\documentclass[11pt]{article}

    \usepackage[breakable]{tcolorbox}
    \usepackage{parskip} % Stop auto-indenting (to mimic markdown behaviour)
    
    \usepackage{iftex}
    \ifPDFTeX
    	\usepackage[T1]{fontenc}
    	\usepackage{mathpazo}
    \else
    	\usepackage{fontspec}
    \fi

    % Basic figure setup, for now with no caption control since it's done
    % automatically by Pandoc (which extracts ![](path) syntax from Markdown).
    \usepackage{graphicx}
    % Maintain compatibility with old templates. Remove in nbconvert 6.0
    \let\Oldincludegraphics\includegraphics
    % Ensure that by default, figures have no caption (until we provide a
    % proper Figure object with a Caption API and a way to capture that
    % in the conversion process - todo).
    \usepackage{caption}
    \DeclareCaptionFormat{nocaption}{}
    \captionsetup{format=nocaption,aboveskip=0pt,belowskip=0pt}

    \usepackage[Export]{adjustbox} % Used to constrain images to a maximum size
    \adjustboxset{max size={0.9\linewidth}{0.9\paperheight}}
    \usepackage{float}
    \floatplacement{figure}{H} % forces figures to be placed at the correct location
    \usepackage{xcolor} % Allow colors to be defined
    \usepackage{enumerate} % Needed for markdown enumerations to work
    \usepackage{geometry} % Used to adjust the document margins
    \usepackage{amsmath} % Equations
    \usepackage{amssymb} % Equations
    \usepackage{textcomp} % defines textquotesingle
    % Hack from http://tex.stackexchange.com/a/47451/13684:
    \AtBeginDocument{%
        \def\PYZsq{\textquotesingle}% Upright quotes in Pygmentized code
    }
    \usepackage{upquote} % Upright quotes for verbatim code
    \usepackage{eurosym} % defines \euro
    \usepackage[mathletters]{ucs} % Extended unicode (utf-8) support
    \usepackage{fancyvrb} % verbatim replacement that allows latex
    \usepackage{grffile} % extends the file name processing of package graphics 
                         % to support a larger range
    \makeatletter % fix for grffile with XeLaTeX
    \def\Gread@@xetex#1{%
      \IfFileExists{"\Gin@base".bb}%
      {\Gread@eps{\Gin@base.bb}}%
      {\Gread@@xetex@aux#1}%
    }
    \makeatother

    % The hyperref package gives us a pdf with properly built
    % internal navigation ('pdf bookmarks' for the table of contents,
    % internal cross-reference links, web links for URLs, etc.)
    \usepackage{hyperref}
    % The default LaTeX title has an obnoxious amount of whitespace. By default,
    % titling removes some of it. It also provides customization options.
    \usepackage{titling}
    \usepackage{longtable} % longtable support required by pandoc >1.10
    \usepackage{booktabs}  % table support for pandoc > 1.12.2
    \usepackage[inline]{enumitem} % IRkernel/repr support (it uses the enumerate* environment)
    \usepackage[normalem]{ulem} % ulem is needed to support strikethroughs (\sout)
                                % normalem makes italics be italics, not underlines
    \usepackage{mathrsfs}
    

    
    % Colors for the hyperref package
    \definecolor{urlcolor}{rgb}{0,.145,.698}
    \definecolor{linkcolor}{rgb}{.71,0.21,0.01}
    \definecolor{citecolor}{rgb}{.12,.54,.11}

    % ANSI colors
    \definecolor{ansi-black}{HTML}{3E424D}
    \definecolor{ansi-black-intense}{HTML}{282C36}
    \definecolor{ansi-red}{HTML}{E75C58}
    \definecolor{ansi-red-intense}{HTML}{B22B31}
    \definecolor{ansi-green}{HTML}{00A250}
    \definecolor{ansi-green-intense}{HTML}{007427}
    \definecolor{ansi-yellow}{HTML}{DDB62B}
    \definecolor{ansi-yellow-intense}{HTML}{B27D12}
    \definecolor{ansi-blue}{HTML}{208FFB}
    \definecolor{ansi-blue-intense}{HTML}{0065CA}
    \definecolor{ansi-magenta}{HTML}{D160C4}
    \definecolor{ansi-magenta-intense}{HTML}{A03196}
    \definecolor{ansi-cyan}{HTML}{60C6C8}
    \definecolor{ansi-cyan-intense}{HTML}{258F8F}
    \definecolor{ansi-white}{HTML}{C5C1B4}
    \definecolor{ansi-white-intense}{HTML}{A1A6B2}
    \definecolor{ansi-default-inverse-fg}{HTML}{FFFFFF}
    \definecolor{ansi-default-inverse-bg}{HTML}{000000}

    % commands and environments needed by pandoc snippets
    % extracted from the output of `pandoc -s`
    \providecommand{\tightlist}{%
      \setlength{\itemsep}{0pt}\setlength{\parskip}{0pt}}
    \DefineVerbatimEnvironment{Highlighting}{Verbatim}{commandchars=\\\{\}}
    % Add ',fontsize=\small' for more characters per line
    \newenvironment{Shaded}{}{}
    \newcommand{\KeywordTok}[1]{\textcolor[rgb]{0.00,0.44,0.13}{\textbf{{#1}}}}
    \newcommand{\DataTypeTok}[1]{\textcolor[rgb]{0.56,0.13,0.00}{{#1}}}
    \newcommand{\DecValTok}[1]{\textcolor[rgb]{0.25,0.63,0.44}{{#1}}}
    \newcommand{\BaseNTok}[1]{\textcolor[rgb]{0.25,0.63,0.44}{{#1}}}
    \newcommand{\FloatTok}[1]{\textcolor[rgb]{0.25,0.63,0.44}{{#1}}}
    \newcommand{\CharTok}[1]{\textcolor[rgb]{0.25,0.44,0.63}{{#1}}}
    \newcommand{\StringTok}[1]{\textcolor[rgb]{0.25,0.44,0.63}{{#1}}}
    \newcommand{\CommentTok}[1]{\textcolor[rgb]{0.38,0.63,0.69}{\textit{{#1}}}}
    \newcommand{\OtherTok}[1]{\textcolor[rgb]{0.00,0.44,0.13}{{#1}}}
    \newcommand{\AlertTok}[1]{\textcolor[rgb]{1.00,0.00,0.00}{\textbf{{#1}}}}
    \newcommand{\FunctionTok}[1]{\textcolor[rgb]{0.02,0.16,0.49}{{#1}}}
    \newcommand{\RegionMarkerTok}[1]{{#1}}
    \newcommand{\ErrorTok}[1]{\textcolor[rgb]{1.00,0.00,0.00}{\textbf{{#1}}}}
    \newcommand{\NormalTok}[1]{{#1}}
    
    % Additional commands for more recent versions of Pandoc
    \newcommand{\ConstantTok}[1]{\textcolor[rgb]{0.53,0.00,0.00}{{#1}}}
    \newcommand{\SpecialCharTok}[1]{\textcolor[rgb]{0.25,0.44,0.63}{{#1}}}
    \newcommand{\VerbatimStringTok}[1]{\textcolor[rgb]{0.25,0.44,0.63}{{#1}}}
    \newcommand{\SpecialStringTok}[1]{\textcolor[rgb]{0.73,0.40,0.53}{{#1}}}
    \newcommand{\ImportTok}[1]{{#1}}
    \newcommand{\DocumentationTok}[1]{\textcolor[rgb]{0.73,0.13,0.13}{\textit{{#1}}}}
    \newcommand{\AnnotationTok}[1]{\textcolor[rgb]{0.38,0.63,0.69}{\textbf{\textit{{#1}}}}}
    \newcommand{\CommentVarTok}[1]{\textcolor[rgb]{0.38,0.63,0.69}{\textbf{\textit{{#1}}}}}
    \newcommand{\VariableTok}[1]{\textcolor[rgb]{0.10,0.09,0.49}{{#1}}}
    \newcommand{\ControlFlowTok}[1]{\textcolor[rgb]{0.00,0.44,0.13}{\textbf{{#1}}}}
    \newcommand{\OperatorTok}[1]{\textcolor[rgb]{0.40,0.40,0.40}{{#1}}}
    \newcommand{\BuiltInTok}[1]{{#1}}
    \newcommand{\ExtensionTok}[1]{{#1}}
    \newcommand{\PreprocessorTok}[1]{\textcolor[rgb]{0.74,0.48,0.00}{{#1}}}
    \newcommand{\AttributeTok}[1]{\textcolor[rgb]{0.49,0.56,0.16}{{#1}}}
    \newcommand{\InformationTok}[1]{\textcolor[rgb]{0.38,0.63,0.69}{\textbf{\textit{{#1}}}}}
    \newcommand{\WarningTok}[1]{\textcolor[rgb]{0.38,0.63,0.69}{\textbf{\textit{{#1}}}}}
    
    
    % Define a nice break command that doesn't care if a line doesn't already
    % exist.
    \def\br{\hspace*{\fill} \\* }
    % Math Jax compatibility definitions
    \def\gt{>}
    \def\lt{<}
    \let\Oldtex\TeX
    \let\Oldlatex\LaTeX
    \renewcommand{\TeX}{\textrm{\Oldtex}}
    \renewcommand{\LaTeX}{\textrm{\Oldlatex}}
    % Document parameters
    % Document title
    \title{tutorial}
    
    
    
    
    
% Pygments definitions
\makeatletter
\def\PY@reset{\let\PY@it=\relax \let\PY@bf=\relax%
    \let\PY@ul=\relax \let\PY@tc=\relax%
    \let\PY@bc=\relax \let\PY@ff=\relax}
\def\PY@tok#1{\csname PY@tok@#1\endcsname}
\def\PY@toks#1+{\ifx\relax#1\empty\else%
    \PY@tok{#1}\expandafter\PY@toks\fi}
\def\PY@do#1{\PY@bc{\PY@tc{\PY@ul{%
    \PY@it{\PY@bf{\PY@ff{#1}}}}}}}
\def\PY#1#2{\PY@reset\PY@toks#1+\relax+\PY@do{#2}}

\expandafter\def\csname PY@tok@w\endcsname{\def\PY@tc##1{\textcolor[rgb]{0.73,0.73,0.73}{##1}}}
\expandafter\def\csname PY@tok@c\endcsname{\let\PY@it=\textit\def\PY@tc##1{\textcolor[rgb]{0.25,0.50,0.50}{##1}}}
\expandafter\def\csname PY@tok@cp\endcsname{\def\PY@tc##1{\textcolor[rgb]{0.74,0.48,0.00}{##1}}}
\expandafter\def\csname PY@tok@k\endcsname{\let\PY@bf=\textbf\def\PY@tc##1{\textcolor[rgb]{0.00,0.50,0.00}{##1}}}
\expandafter\def\csname PY@tok@kp\endcsname{\def\PY@tc##1{\textcolor[rgb]{0.00,0.50,0.00}{##1}}}
\expandafter\def\csname PY@tok@kt\endcsname{\def\PY@tc##1{\textcolor[rgb]{0.69,0.00,0.25}{##1}}}
\expandafter\def\csname PY@tok@o\endcsname{\def\PY@tc##1{\textcolor[rgb]{0.40,0.40,0.40}{##1}}}
\expandafter\def\csname PY@tok@ow\endcsname{\let\PY@bf=\textbf\def\PY@tc##1{\textcolor[rgb]{0.67,0.13,1.00}{##1}}}
\expandafter\def\csname PY@tok@nb\endcsname{\def\PY@tc##1{\textcolor[rgb]{0.00,0.50,0.00}{##1}}}
\expandafter\def\csname PY@tok@nf\endcsname{\def\PY@tc##1{\textcolor[rgb]{0.00,0.00,1.00}{##1}}}
\expandafter\def\csname PY@tok@nc\endcsname{\let\PY@bf=\textbf\def\PY@tc##1{\textcolor[rgb]{0.00,0.00,1.00}{##1}}}
\expandafter\def\csname PY@tok@nn\endcsname{\let\PY@bf=\textbf\def\PY@tc##1{\textcolor[rgb]{0.00,0.00,1.00}{##1}}}
\expandafter\def\csname PY@tok@ne\endcsname{\let\PY@bf=\textbf\def\PY@tc##1{\textcolor[rgb]{0.82,0.25,0.23}{##1}}}
\expandafter\def\csname PY@tok@nv\endcsname{\def\PY@tc##1{\textcolor[rgb]{0.10,0.09,0.49}{##1}}}
\expandafter\def\csname PY@tok@no\endcsname{\def\PY@tc##1{\textcolor[rgb]{0.53,0.00,0.00}{##1}}}
\expandafter\def\csname PY@tok@nl\endcsname{\def\PY@tc##1{\textcolor[rgb]{0.63,0.63,0.00}{##1}}}
\expandafter\def\csname PY@tok@ni\endcsname{\let\PY@bf=\textbf\def\PY@tc##1{\textcolor[rgb]{0.60,0.60,0.60}{##1}}}
\expandafter\def\csname PY@tok@na\endcsname{\def\PY@tc##1{\textcolor[rgb]{0.49,0.56,0.16}{##1}}}
\expandafter\def\csname PY@tok@nt\endcsname{\let\PY@bf=\textbf\def\PY@tc##1{\textcolor[rgb]{0.00,0.50,0.00}{##1}}}
\expandafter\def\csname PY@tok@nd\endcsname{\def\PY@tc##1{\textcolor[rgb]{0.67,0.13,1.00}{##1}}}
\expandafter\def\csname PY@tok@s\endcsname{\def\PY@tc##1{\textcolor[rgb]{0.73,0.13,0.13}{##1}}}
\expandafter\def\csname PY@tok@sd\endcsname{\let\PY@it=\textit\def\PY@tc##1{\textcolor[rgb]{0.73,0.13,0.13}{##1}}}
\expandafter\def\csname PY@tok@si\endcsname{\let\PY@bf=\textbf\def\PY@tc##1{\textcolor[rgb]{0.73,0.40,0.53}{##1}}}
\expandafter\def\csname PY@tok@se\endcsname{\let\PY@bf=\textbf\def\PY@tc##1{\textcolor[rgb]{0.73,0.40,0.13}{##1}}}
\expandafter\def\csname PY@tok@sr\endcsname{\def\PY@tc##1{\textcolor[rgb]{0.73,0.40,0.53}{##1}}}
\expandafter\def\csname PY@tok@ss\endcsname{\def\PY@tc##1{\textcolor[rgb]{0.10,0.09,0.49}{##1}}}
\expandafter\def\csname PY@tok@sx\endcsname{\def\PY@tc##1{\textcolor[rgb]{0.00,0.50,0.00}{##1}}}
\expandafter\def\csname PY@tok@m\endcsname{\def\PY@tc##1{\textcolor[rgb]{0.40,0.40,0.40}{##1}}}
\expandafter\def\csname PY@tok@gh\endcsname{\let\PY@bf=\textbf\def\PY@tc##1{\textcolor[rgb]{0.00,0.00,0.50}{##1}}}
\expandafter\def\csname PY@tok@gu\endcsname{\let\PY@bf=\textbf\def\PY@tc##1{\textcolor[rgb]{0.50,0.00,0.50}{##1}}}
\expandafter\def\csname PY@tok@gd\endcsname{\def\PY@tc##1{\textcolor[rgb]{0.63,0.00,0.00}{##1}}}
\expandafter\def\csname PY@tok@gi\endcsname{\def\PY@tc##1{\textcolor[rgb]{0.00,0.63,0.00}{##1}}}
\expandafter\def\csname PY@tok@gr\endcsname{\def\PY@tc##1{\textcolor[rgb]{1.00,0.00,0.00}{##1}}}
\expandafter\def\csname PY@tok@ge\endcsname{\let\PY@it=\textit}
\expandafter\def\csname PY@tok@gs\endcsname{\let\PY@bf=\textbf}
\expandafter\def\csname PY@tok@gp\endcsname{\let\PY@bf=\textbf\def\PY@tc##1{\textcolor[rgb]{0.00,0.00,0.50}{##1}}}
\expandafter\def\csname PY@tok@go\endcsname{\def\PY@tc##1{\textcolor[rgb]{0.53,0.53,0.53}{##1}}}
\expandafter\def\csname PY@tok@gt\endcsname{\def\PY@tc##1{\textcolor[rgb]{0.00,0.27,0.87}{##1}}}
\expandafter\def\csname PY@tok@err\endcsname{\def\PY@bc##1{\setlength{\fboxsep}{0pt}\fcolorbox[rgb]{1.00,0.00,0.00}{1,1,1}{\strut ##1}}}
\expandafter\def\csname PY@tok@kc\endcsname{\let\PY@bf=\textbf\def\PY@tc##1{\textcolor[rgb]{0.00,0.50,0.00}{##1}}}
\expandafter\def\csname PY@tok@kd\endcsname{\let\PY@bf=\textbf\def\PY@tc##1{\textcolor[rgb]{0.00,0.50,0.00}{##1}}}
\expandafter\def\csname PY@tok@kn\endcsname{\let\PY@bf=\textbf\def\PY@tc##1{\textcolor[rgb]{0.00,0.50,0.00}{##1}}}
\expandafter\def\csname PY@tok@kr\endcsname{\let\PY@bf=\textbf\def\PY@tc##1{\textcolor[rgb]{0.00,0.50,0.00}{##1}}}
\expandafter\def\csname PY@tok@bp\endcsname{\def\PY@tc##1{\textcolor[rgb]{0.00,0.50,0.00}{##1}}}
\expandafter\def\csname PY@tok@fm\endcsname{\def\PY@tc##1{\textcolor[rgb]{0.00,0.00,1.00}{##1}}}
\expandafter\def\csname PY@tok@vc\endcsname{\def\PY@tc##1{\textcolor[rgb]{0.10,0.09,0.49}{##1}}}
\expandafter\def\csname PY@tok@vg\endcsname{\def\PY@tc##1{\textcolor[rgb]{0.10,0.09,0.49}{##1}}}
\expandafter\def\csname PY@tok@vi\endcsname{\def\PY@tc##1{\textcolor[rgb]{0.10,0.09,0.49}{##1}}}
\expandafter\def\csname PY@tok@vm\endcsname{\def\PY@tc##1{\textcolor[rgb]{0.10,0.09,0.49}{##1}}}
\expandafter\def\csname PY@tok@sa\endcsname{\def\PY@tc##1{\textcolor[rgb]{0.73,0.13,0.13}{##1}}}
\expandafter\def\csname PY@tok@sb\endcsname{\def\PY@tc##1{\textcolor[rgb]{0.73,0.13,0.13}{##1}}}
\expandafter\def\csname PY@tok@sc\endcsname{\def\PY@tc##1{\textcolor[rgb]{0.73,0.13,0.13}{##1}}}
\expandafter\def\csname PY@tok@dl\endcsname{\def\PY@tc##1{\textcolor[rgb]{0.73,0.13,0.13}{##1}}}
\expandafter\def\csname PY@tok@s2\endcsname{\def\PY@tc##1{\textcolor[rgb]{0.73,0.13,0.13}{##1}}}
\expandafter\def\csname PY@tok@sh\endcsname{\def\PY@tc##1{\textcolor[rgb]{0.73,0.13,0.13}{##1}}}
\expandafter\def\csname PY@tok@s1\endcsname{\def\PY@tc##1{\textcolor[rgb]{0.73,0.13,0.13}{##1}}}
\expandafter\def\csname PY@tok@mb\endcsname{\def\PY@tc##1{\textcolor[rgb]{0.40,0.40,0.40}{##1}}}
\expandafter\def\csname PY@tok@mf\endcsname{\def\PY@tc##1{\textcolor[rgb]{0.40,0.40,0.40}{##1}}}
\expandafter\def\csname PY@tok@mh\endcsname{\def\PY@tc##1{\textcolor[rgb]{0.40,0.40,0.40}{##1}}}
\expandafter\def\csname PY@tok@mi\endcsname{\def\PY@tc##1{\textcolor[rgb]{0.40,0.40,0.40}{##1}}}
\expandafter\def\csname PY@tok@il\endcsname{\def\PY@tc##1{\textcolor[rgb]{0.40,0.40,0.40}{##1}}}
\expandafter\def\csname PY@tok@mo\endcsname{\def\PY@tc##1{\textcolor[rgb]{0.40,0.40,0.40}{##1}}}
\expandafter\def\csname PY@tok@ch\endcsname{\let\PY@it=\textit\def\PY@tc##1{\textcolor[rgb]{0.25,0.50,0.50}{##1}}}
\expandafter\def\csname PY@tok@cm\endcsname{\let\PY@it=\textit\def\PY@tc##1{\textcolor[rgb]{0.25,0.50,0.50}{##1}}}
\expandafter\def\csname PY@tok@cpf\endcsname{\let\PY@it=\textit\def\PY@tc##1{\textcolor[rgb]{0.25,0.50,0.50}{##1}}}
\expandafter\def\csname PY@tok@c1\endcsname{\let\PY@it=\textit\def\PY@tc##1{\textcolor[rgb]{0.25,0.50,0.50}{##1}}}
\expandafter\def\csname PY@tok@cs\endcsname{\let\PY@it=\textit\def\PY@tc##1{\textcolor[rgb]{0.25,0.50,0.50}{##1}}}

\def\PYZbs{\char`\\}
\def\PYZus{\char`\_}
\def\PYZob{\char`\{}
\def\PYZcb{\char`\}}
\def\PYZca{\char`\^}
\def\PYZam{\char`\&}
\def\PYZlt{\char`\<}
\def\PYZgt{\char`\>}
\def\PYZsh{\char`\#}
\def\PYZpc{\char`\%}
\def\PYZdl{\char`\$}
\def\PYZhy{\char`\-}
\def\PYZsq{\char`\'}
\def\PYZdq{\char`\"}
\def\PYZti{\char`\~}
% for compatibility with earlier versions
\def\PYZat{@}
\def\PYZlb{[}
\def\PYZrb{]}
\makeatother


    % For linebreaks inside Verbatim environment from package fancyvrb. 
    \makeatletter
        \newbox\Wrappedcontinuationbox 
        \newbox\Wrappedvisiblespacebox 
        \newcommand*\Wrappedvisiblespace {\textcolor{red}{\textvisiblespace}} 
        \newcommand*\Wrappedcontinuationsymbol {\textcolor{red}{\llap{\tiny$\m@th\hookrightarrow$}}} 
        \newcommand*\Wrappedcontinuationindent {3ex } 
        \newcommand*\Wrappedafterbreak {\kern\Wrappedcontinuationindent\copy\Wrappedcontinuationbox} 
        % Take advantage of the already applied Pygments mark-up to insert 
        % potential linebreaks for TeX processing. 
        %        {, <, #, %, $, ' and ": go to next line. 
        %        _, }, ^, &, >, - and ~: stay at end of broken line. 
        % Use of \textquotesingle for straight quote. 
        \newcommand*\Wrappedbreaksatspecials {% 
            \def\PYGZus{\discretionary{\char`\_}{\Wrappedafterbreak}{\char`\_}}% 
            \def\PYGZob{\discretionary{}{\Wrappedafterbreak\char`\{}{\char`\{}}% 
            \def\PYGZcb{\discretionary{\char`\}}{\Wrappedafterbreak}{\char`\}}}% 
            \def\PYGZca{\discretionary{\char`\^}{\Wrappedafterbreak}{\char`\^}}% 
            \def\PYGZam{\discretionary{\char`\&}{\Wrappedafterbreak}{\char`\&}}% 
            \def\PYGZlt{\discretionary{}{\Wrappedafterbreak\char`\<}{\char`\<}}% 
            \def\PYGZgt{\discretionary{\char`\>}{\Wrappedafterbreak}{\char`\>}}% 
            \def\PYGZsh{\discretionary{}{\Wrappedafterbreak\char`\#}{\char`\#}}% 
            \def\PYGZpc{\discretionary{}{\Wrappedafterbreak\char`\%}{\char`\%}}% 
            \def\PYGZdl{\discretionary{}{\Wrappedafterbreak\char`\$}{\char`\$}}% 
            \def\PYGZhy{\discretionary{\char`\-}{\Wrappedafterbreak}{\char`\-}}% 
            \def\PYGZsq{\discretionary{}{\Wrappedafterbreak\textquotesingle}{\textquotesingle}}% 
            \def\PYGZdq{\discretionary{}{\Wrappedafterbreak\char`\"}{\char`\"}}% 
            \def\PYGZti{\discretionary{\char`\~}{\Wrappedafterbreak}{\char`\~}}% 
        } 
        % Some characters . , ; ? ! / are not pygmentized. 
        % This macro makes them "active" and they will insert potential linebreaks 
        \newcommand*\Wrappedbreaksatpunct {% 
            \lccode`\~`\.\lowercase{\def~}{\discretionary{\hbox{\char`\.}}{\Wrappedafterbreak}{\hbox{\char`\.}}}% 
            \lccode`\~`\,\lowercase{\def~}{\discretionary{\hbox{\char`\,}}{\Wrappedafterbreak}{\hbox{\char`\,}}}% 
            \lccode`\~`\;\lowercase{\def~}{\discretionary{\hbox{\char`\;}}{\Wrappedafterbreak}{\hbox{\char`\;}}}% 
            \lccode`\~`\:\lowercase{\def~}{\discretionary{\hbox{\char`\:}}{\Wrappedafterbreak}{\hbox{\char`\:}}}% 
            \lccode`\~`\?\lowercase{\def~}{\discretionary{\hbox{\char`\?}}{\Wrappedafterbreak}{\hbox{\char`\?}}}% 
            \lccode`\~`\!\lowercase{\def~}{\discretionary{\hbox{\char`\!}}{\Wrappedafterbreak}{\hbox{\char`\!}}}% 
            \lccode`\~`\/\lowercase{\def~}{\discretionary{\hbox{\char`\/}}{\Wrappedafterbreak}{\hbox{\char`\/}}}% 
            \catcode`\.\active
            \catcode`\,\active 
            \catcode`\;\active
            \catcode`\:\active
            \catcode`\?\active
            \catcode`\!\active
            \catcode`\/\active 
            \lccode`\~`\~ 	
        }
    \makeatother

    \let\OriginalVerbatim=\Verbatim
    \makeatletter
    \renewcommand{\Verbatim}[1][1]{%
        %\parskip\z@skip
        \sbox\Wrappedcontinuationbox {\Wrappedcontinuationsymbol}%
        \sbox\Wrappedvisiblespacebox {\FV@SetupFont\Wrappedvisiblespace}%
        \def\FancyVerbFormatLine ##1{\hsize\linewidth
            \vtop{\raggedright\hyphenpenalty\z@\exhyphenpenalty\z@
                \doublehyphendemerits\z@\finalhyphendemerits\z@
                \strut ##1\strut}%
        }%
        % If the linebreak is at a space, the latter will be displayed as visible
        % space at end of first line, and a continuation symbol starts next line.
        % Stretch/shrink are however usually zero for typewriter font.
        \def\FV@Space {%
            \nobreak\hskip\z@ plus\fontdimen3\font minus\fontdimen4\font
            \discretionary{\copy\Wrappedvisiblespacebox}{\Wrappedafterbreak}
            {\kern\fontdimen2\font}%
        }%
        
        % Allow breaks at special characters using \PYG... macros.
        \Wrappedbreaksatspecials
        % Breaks at punctuation characters . , ; ? ! and / need catcode=\active 	
        \OriginalVerbatim[#1,codes*=\Wrappedbreaksatpunct]%
    }
    \makeatother

    % Exact colors from NB
    \definecolor{incolor}{HTML}{303F9F}
    \definecolor{outcolor}{HTML}{D84315}
    \definecolor{cellborder}{HTML}{CFCFCF}
    \definecolor{cellbackground}{HTML}{F7F7F7}
    
    % prompt
    \makeatletter
    \newcommand{\boxspacing}{\kern\kvtcb@left@rule\kern\kvtcb@boxsep}
    \makeatother
    \newcommand{\prompt}[4]{
        \ttfamily\llap{{\color{#2}[#3]:\hspace{3pt}#4}}\vspace{-\baselineskip}
    }
    

    
    % Prevent overflowing lines due to hard-to-break entities
    \sloppy 
    % Setup hyperref package
    \hypersetup{
      breaklinks=true,  % so long urls are correctly broken across lines
      colorlinks=true,
      urlcolor=urlcolor,
      linkcolor=linkcolor,
      citecolor=citecolor,
      }
    % Slightly bigger margins than the latex defaults
    
    \geometry{verbose,tmargin=1in,bmargin=1in,lmargin=1in,rmargin=1in}
    
    

\begin{document}
    
    \maketitle
    
    

    
    \section{Introduction}\label{introduction}

    \subsection{Disclaimer}\label{disclaimer}

This is inspired from Dr. Andrew Gelman's case study, which can be found
\href{https://mc-stan.org/users/documentation/case-studies/golf.html}{here}.
Specifically:

\begin{itemize}
\item
  \begin{itemize}
  \tightlist
  \item
    This is heavily inspired by Colin Caroll's Blog present
    \href{https://nbviewer.jupyter.org/github/pymc-devs/pymc3/blob/master/docs/source/notebooks/putting_workflow.ipynb}{here}.
    A lot of the plotting code from his blog post has been reused.
  \end{itemize}
\item
  Josh Duncan's blog post on the same topic which can be found
  \href{https://jduncstats.com/post/2019-11-02_golf-turing/}{here}.
\end{itemize}

This is not a novel solution. It is merely a replication of Dr. Gelman's
blog in PyMC3.

    \subsection{Problem}\label{problem}

    This is based on a popular blog post by Dr. Andrew Gelman. Here, we are
given data from professional golfers on the proportion of success putts
from a number of tries. Our aim is to identify:

\begin{quote}
Can we model the probability of success in golf putting as a function of
distance from the hole?
\end{quote}

    \subsection{EDA}\label{eda}

    \begin{tcolorbox}[breakable, size=fbox, boxrule=1pt, pad at break*=1mm,colback=cellbackground, colframe=cellborder]
\prompt{In}{incolor}{1}{\boxspacing}
\begin{Verbatim}[commandchars=\\\{\}]
\PY{k+kn}{import} \PY{n+nn}{pandas} \PY{k}{as} \PY{n+nn}{pd}
\PY{k+kn}{import} \PY{n+nn}{numpy} \PY{k}{as} \PY{n+nn}{np}
\PY{k+kn}{import} \PY{n+nn}{pymc3} \PY{k}{as} \PY{n+nn}{pm}
\PY{k+kn}{import} \PY{n+nn}{matplotlib}\PY{n+nn}{.}\PY{n+nn}{pyplot} \PY{k}{as} \PY{n+nn}{plt}
\PY{k+kn}{import} \PY{n+nn}{seaborn} \PY{k}{as} \PY{n+nn}{sns}
\end{Verbatim}
\end{tcolorbox}

    \begin{Verbatim}[commandchars=\\\{\}]
WARNING (theano.tensor.blas): Using NumPy C-API based implementation for BLAS
functions.
    \end{Verbatim}

    The source repository is present
\href{https://github.com/stan-dev/example-models/tree/master/knitr/golf}{here}

    \begin{tcolorbox}[breakable, size=fbox, boxrule=1pt, pad at break*=1mm,colback=cellbackground, colframe=cellborder]
\prompt{In}{incolor}{2}{\boxspacing}
\begin{Verbatim}[commandchars=\\\{\}]
\PY{n}{data} \PY{o}{=} \PY{n}{np}\PY{o}{.}\PY{n}{array}\PY{p}{(}\PY{p}{[}\PY{p}{[}\PY{l+m+mi}{2}\PY{p}{,}\PY{l+m+mi}{1443}\PY{p}{,}\PY{l+m+mi}{1346}\PY{p}{]}\PY{p}{,}
\PY{p}{[}\PY{l+m+mi}{3}\PY{p}{,}\PY{l+m+mi}{694}\PY{p}{,}\PY{l+m+mi}{577}\PY{p}{]}\PY{p}{,}
\PY{p}{[}\PY{l+m+mi}{4}\PY{p}{,}\PY{l+m+mi}{455}\PY{p}{,}\PY{l+m+mi}{337}\PY{p}{]}\PY{p}{,}
\PY{p}{[}\PY{l+m+mi}{5}\PY{p}{,}\PY{l+m+mi}{353}\PY{p}{,}\PY{l+m+mi}{208}\PY{p}{]}\PY{p}{,}
\PY{p}{[}\PY{l+m+mi}{6}\PY{p}{,}\PY{l+m+mi}{272}\PY{p}{,}\PY{l+m+mi}{149}\PY{p}{]}\PY{p}{,}
\PY{p}{[}\PY{l+m+mi}{7}\PY{p}{,}\PY{l+m+mi}{256}\PY{p}{,}\PY{l+m+mi}{136}\PY{p}{]}\PY{p}{,}
\PY{p}{[}\PY{l+m+mi}{8}\PY{p}{,}\PY{l+m+mi}{240}\PY{p}{,}\PY{l+m+mi}{111}\PY{p}{]}\PY{p}{,}
\PY{p}{[}\PY{l+m+mi}{9}\PY{p}{,}\PY{l+m+mi}{217}\PY{p}{,}\PY{l+m+mi}{69}\PY{p}{]}\PY{p}{,}
\PY{p}{[}\PY{l+m+mi}{10}\PY{p}{,}\PY{l+m+mi}{200}\PY{p}{,}\PY{l+m+mi}{67}\PY{p}{]}\PY{p}{,}
\PY{p}{[}\PY{l+m+mi}{11}\PY{p}{,}\PY{l+m+mi}{237}\PY{p}{,}\PY{l+m+mi}{75}\PY{p}{]}\PY{p}{,}
\PY{p}{[}\PY{l+m+mi}{12}\PY{p}{,}\PY{l+m+mi}{202}\PY{p}{,}\PY{l+m+mi}{52}\PY{p}{]}\PY{p}{,}
\PY{p}{[}\PY{l+m+mi}{13}\PY{p}{,}\PY{l+m+mi}{192}\PY{p}{,}\PY{l+m+mi}{46}\PY{p}{]}\PY{p}{,}
\PY{p}{[}\PY{l+m+mi}{14}\PY{p}{,}\PY{l+m+mi}{174}\PY{p}{,}\PY{l+m+mi}{54}\PY{p}{]}\PY{p}{,}
\PY{p}{[}\PY{l+m+mi}{15}\PY{p}{,}\PY{l+m+mi}{167}\PY{p}{,}\PY{l+m+mi}{28}\PY{p}{]}\PY{p}{,}
\PY{p}{[}\PY{l+m+mi}{16}\PY{p}{,}\PY{l+m+mi}{201}\PY{p}{,}\PY{l+m+mi}{27}\PY{p}{]}\PY{p}{,}
\PY{p}{[}\PY{l+m+mi}{17}\PY{p}{,}\PY{l+m+mi}{195}\PY{p}{,}\PY{l+m+mi}{31}\PY{p}{]}\PY{p}{,}
\PY{p}{[}\PY{l+m+mi}{18}\PY{p}{,}\PY{l+m+mi}{191}\PY{p}{,}\PY{l+m+mi}{33}\PY{p}{]}\PY{p}{,}
\PY{p}{[}\PY{l+m+mi}{19}\PY{p}{,}\PY{l+m+mi}{147}\PY{p}{,}\PY{l+m+mi}{20}\PY{p}{]}\PY{p}{,}
\PY{p}{[}\PY{l+m+mi}{20}\PY{p}{,}\PY{l+m+mi}{152}\PY{p}{,}\PY{l+m+mi}{24}\PY{p}{]}\PY{p}{]}\PY{p}{)}

\PY{n}{df} \PY{o}{=} \PY{n}{pd}\PY{o}{.}\PY{n}{DataFrame}\PY{p}{(}\PY{n}{data}\PY{p}{,} \PY{n}{columns}\PY{o}{=}\PY{p}{[}
    \PY{l+s+s1}{\PYZsq{}}\PY{l+s+s1}{distance}\PY{l+s+s1}{\PYZsq{}}\PY{p}{,} 
    \PY{l+s+s1}{\PYZsq{}}\PY{l+s+s1}{tries}\PY{l+s+s1}{\PYZsq{}}\PY{p}{,} 
    \PY{l+s+s1}{\PYZsq{}}\PY{l+s+s1}{success\PYZus{}count}\PY{l+s+s1}{\PYZsq{}}
\PY{p}{]}\PY{p}{)}
\end{Verbatim}
\end{tcolorbox}

    \begin{tcolorbox}[breakable, size=fbox, boxrule=1pt, pad at break*=1mm,colback=cellbackground, colframe=cellborder]
\prompt{In}{incolor}{3}{\boxspacing}
\begin{Verbatim}[commandchars=\\\{\}]
\PY{n}{df}
\end{Verbatim}
\end{tcolorbox}

            \begin{tcolorbox}[breakable, size=fbox, boxrule=.5pt, pad at break*=1mm, opacityfill=0]
\prompt{Out}{outcolor}{3}{\boxspacing}
\begin{Verbatim}[commandchars=\\\{\}]
    distance  tries  success\_count
0          2   1443           1346
1          3    694            577
2          4    455            337
3          5    353            208
4          6    272            149
5          7    256            136
6          8    240            111
7          9    217             69
8         10    200             67
9         11    237             75
10        12    202             52
11        13    192             46
12        14    174             54
13        15    167             28
14        16    201             27
15        17    195             31
16        18    191             33
17        19    147             20
18        20    152             24
\end{Verbatim}
\end{tcolorbox}
        
    The variables have the following format:

\begin{longtable}[]{@{}lll@{}}
\toprule
Variable & Units & Description\tabularnewline
\midrule
\endhead
distance & feet & Distance from the hole for the putt
attempt\tabularnewline
tries & count & Number of attempts at the chosen distance\tabularnewline
success\_count & count & The total successful putts\tabularnewline
\bottomrule
\end{longtable}

    Lets try to visualize the dataset:

    \begin{tcolorbox}[breakable, size=fbox, boxrule=1pt, pad at break*=1mm,colback=cellbackground, colframe=cellborder]
\prompt{In}{incolor}{4}{\boxspacing}
\begin{Verbatim}[commandchars=\\\{\}]
\PY{n}{df}\PY{p}{[}\PY{l+s+s1}{\PYZsq{}}\PY{l+s+s1}{success\PYZus{}prob}\PY{l+s+s1}{\PYZsq{}}\PY{p}{]} \PY{o}{=} \PY{n}{df}\PY{o}{.}\PY{n}{success\PYZus{}count} \PY{o}{/} \PY{n}{df}\PY{o}{.}\PY{n}{tries}
\end{Verbatim}
\end{tcolorbox}

    \begin{tcolorbox}[breakable, size=fbox, boxrule=1pt, pad at break*=1mm,colback=cellbackground, colframe=cellborder]
\prompt{In}{incolor}{5}{\boxspacing}
\begin{Verbatim}[commandchars=\\\{\}]
\PY{n}{sns}\PY{o}{.}\PY{n}{set}\PY{p}{(}\PY{p}{)}
\PY{n}{plt}\PY{o}{.}\PY{n}{figure}\PY{p}{(}\PY{n}{figsize}\PY{o}{=}\PY{p}{(}\PY{l+m+mi}{16}\PY{p}{,} \PY{l+m+mi}{6}\PY{p}{)}\PY{p}{)}
\PY{n}{ax} \PY{o}{=} \PY{n}{sns}\PY{o}{.}\PY{n}{scatterplot}\PY{p}{(}\PY{n}{x}\PY{o}{=}\PY{l+s+s1}{\PYZsq{}}\PY{l+s+s1}{distance}\PY{l+s+s1}{\PYZsq{}}\PY{p}{,} \PY{n}{y}\PY{o}{=}\PY{l+s+s1}{\PYZsq{}}\PY{l+s+s1}{success\PYZus{}prob}\PY{l+s+s1}{\PYZsq{}}\PY{p}{,} \PY{n}{data}\PY{o}{=}\PY{n}{df}\PY{p}{,} \PY{n}{s}\PY{o}{=}\PY{l+m+mi}{200}\PY{p}{)}
\PY{n}{ax}\PY{o}{.}\PY{n}{set}\PY{p}{(}\PY{n}{xlabel}\PY{o}{=}\PY{l+s+s1}{\PYZsq{}}\PY{l+s+s1}{Distance from hole(ft)}\PY{l+s+s1}{\PYZsq{}}\PY{p}{,} \PY{n}{ylabel}\PY{o}{=}\PY{l+s+s1}{\PYZsq{}}\PY{l+s+s1}{Probability of Success}\PY{l+s+s1}{\PYZsq{}}\PY{p}{)}
\end{Verbatim}
\end{tcolorbox}

            \begin{tcolorbox}[breakable, size=fbox, boxrule=.5pt, pad at break*=1mm, opacityfill=0]
\prompt{Out}{outcolor}{5}{\boxspacing}
\begin{Verbatim}[commandchars=\\\{\}]
[Text(0, 0.5, 'Probability of Success'),
 Text(0.5, 0, 'Distance from hole(ft)')]
\end{Verbatim}
\end{tcolorbox}
        
    \begin{center}
    \adjustimage{max size={0.9\linewidth}{0.9\paperheight}}{output_12_1.png}
    \end{center}
    { \hspace*{\fill} \\}
    
    We can notice that the \textbf{probability of success decreases as the
distance increases.}

    \section{Baseline Model}\label{baseline-model}

    Let us try to see we can fit a simple linear model to the data i.e
Logsitic Regression. We will be using PyMC3.

    Here, we will attempt to model the success of golf putting by using the
distance as an independant(i.e predictor) variable. The model will have
the following form:

\textbf{\[y_i \sim binomial(n_j, logit^{-1}(b_0 + b_1x_j)), \text{for } j = 1,...J \]}

    \begin{tcolorbox}[breakable, size=fbox, boxrule=1pt, pad at break*=1mm,colback=cellbackground, colframe=cellborder]
\prompt{In}{incolor}{6}{\boxspacing}
\begin{Verbatim}[commandchars=\\\{\}]
\PY{k}{with} \PY{n}{pm}\PY{o}{.}\PY{n}{Model}\PY{p}{(}\PY{p}{)} \PY{k}{as} \PY{n}{model}\PY{p}{:}
    \PY{n}{b\PYZus{}0} \PY{o}{=} \PY{n}{pm}\PY{o}{.}\PY{n}{Normal}\PY{p}{(}\PY{l+s+s1}{\PYZsq{}}\PY{l+s+s1}{b\PYZus{}0}\PY{l+s+s1}{\PYZsq{}}\PY{p}{,} \PY{n}{mu}\PY{o}{=}\PY{l+m+mi}{0}\PY{p}{,} \PY{n}{sd}\PY{o}{=}\PY{l+m+mi}{1}\PY{p}{)}
    \PY{n}{b\PYZus{}1} \PY{o}{=} \PY{n}{pm}\PY{o}{.}\PY{n}{Normal}\PY{p}{(}\PY{l+s+s1}{\PYZsq{}}\PY{l+s+s1}{b\PYZus{}1}\PY{l+s+s1}{\PYZsq{}}\PY{p}{,} \PY{n}{mu}\PY{o}{=}\PY{l+m+mi}{0}\PY{p}{,} \PY{n}{sd}\PY{o}{=}\PY{l+m+mi}{1}\PY{p}{)}
        
    \PY{n}{y} \PY{o}{=} \PY{n}{pm}\PY{o}{.}\PY{n}{Binomial}\PY{p}{(}
        \PY{l+s+s1}{\PYZsq{}}\PY{l+s+s1}{y}\PY{l+s+s1}{\PYZsq{}}\PY{p}{,} 
        \PY{n}{n}\PY{o}{=}\PY{n}{df}\PY{o}{.}\PY{n}{tries}\PY{p}{,} 
        \PY{n}{p}\PY{o}{=}\PY{n}{pm}\PY{o}{.}\PY{n}{math}\PY{o}{.}\PY{n}{invlogit}\PY{p}{(}\PY{n}{b\PYZus{}0} \PY{o}{+} \PY{n}{b\PYZus{}1} \PY{o}{*} \PY{n}{df}\PY{o}{.}\PY{n}{distance}\PY{p}{)}\PY{p}{,} 
        \PY{n}{observed}\PY{o}{=}\PY{n}{df}\PY{o}{.}\PY{n}{success\PYZus{}count}
    \PY{p}{)}
\end{Verbatim}
\end{tcolorbox}

    Why are we using inverse logit?

\begin{itemize}
\tightlist
\item
  \textbf{Logit} is a function used to convert a continous variable to a
  value in the range {[}0,1{]}
\item
  \textbf{Inverse Logit}: Used to convert real valued variable to a
  value in the range {[}0,1{]}
\end{itemize}

    \begin{tcolorbox}[breakable, size=fbox, boxrule=1pt, pad at break*=1mm,colback=cellbackground, colframe=cellborder]
\prompt{In}{incolor}{7}{\boxspacing}
\begin{Verbatim}[commandchars=\\\{\}]
\PY{n}{pm}\PY{o}{.}\PY{n}{model\PYZus{}to\PYZus{}graphviz}\PY{p}{(}\PY{n}{model}\PY{p}{)}
\end{Verbatim}
\end{tcolorbox}
 
            
\prompt{Out}{outcolor}{7}{}
    
    \begin{center}
    \adjustimage{max size={0.9\linewidth}{0.9\paperheight}}{output_19_0.pdf}
    \end{center}
    { \hspace*{\fill} \\}
    

    \begin{tcolorbox}[breakable, size=fbox, boxrule=1pt, pad at break*=1mm,colback=cellbackground, colframe=cellborder]
\prompt{In}{incolor}{8}{\boxspacing}
\begin{Verbatim}[commandchars=\\\{\}]
\PY{k}{with} \PY{n}{model}\PY{p}{:}
    \PY{n}{trace} \PY{o}{=} \PY{n}{pm}\PY{o}{.}\PY{n}{sample}\PY{p}{(}\PY{l+m+mi}{1000}\PY{p}{,} \PY{n}{tune}\PY{o}{=}\PY{l+m+mi}{1000}\PY{p}{,} \PY{n}{chains}\PY{o}{=}\PY{l+m+mi}{4}\PY{p}{)}
\end{Verbatim}
\end{tcolorbox}

    \begin{Verbatim}[commandchars=\\\{\}]
Auto-assigning NUTS sampler{\ldots}
Initializing NUTS using jitter+adapt\_diag{\ldots}
Multiprocess sampling (4 chains in 2 jobs)
NUTS: [b\_1, b\_0]
Sampling 4 chains, 0 divergences: 100\%|██████████| 8000/8000 [00:06<00:00,
1164.30draws/s]
The number of effective samples is smaller than 25\% for some parameters.
    \end{Verbatim}

    \begin{tcolorbox}[breakable, size=fbox, boxrule=1pt, pad at break*=1mm,colback=cellbackground, colframe=cellborder]
\prompt{In}{incolor}{9}{\boxspacing}
\begin{Verbatim}[commandchars=\\\{\}]
\PY{n}{pm}\PY{o}{.}\PY{n}{summary}\PY{p}{(}\PY{n}{trace}\PY{p}{)}\PY{p}{[}\PY{p}{[}\PY{l+s+s1}{\PYZsq{}}\PY{l+s+s1}{mean}\PY{l+s+s1}{\PYZsq{}}\PY{p}{,} \PY{l+s+s1}{\PYZsq{}}\PY{l+s+s1}{sd}\PY{l+s+s1}{\PYZsq{}}\PY{p}{,} \PY{l+s+s1}{\PYZsq{}}\PY{l+s+s1}{mcse\PYZus{}mean}\PY{l+s+s1}{\PYZsq{}}\PY{p}{,} \PY{l+s+s1}{\PYZsq{}}\PY{l+s+s1}{mcse\PYZus{}sd}\PY{l+s+s1}{\PYZsq{}}\PY{p}{,} \PY{l+s+s1}{\PYZsq{}}\PY{l+s+s1}{ess\PYZus{}mean}\PY{l+s+s1}{\PYZsq{}}\PY{p}{,} \PY{l+s+s1}{\PYZsq{}}\PY{l+s+s1}{r\PYZus{}hat}\PY{l+s+s1}{\PYZsq{}}\PY{p}{]}\PY{p}{]}
\end{Verbatim}
\end{tcolorbox}

            \begin{tcolorbox}[breakable, size=fbox, boxrule=.5pt, pad at break*=1mm, opacityfill=0]
\prompt{Out}{outcolor}{9}{\boxspacing}
\begin{Verbatim}[commandchars=\\\{\}]
      mean     sd  mcse\_mean  mcse\_sd  ess\_mean  r\_hat
b\_0  2.223  0.058      0.002    0.001     997.0    1.0
b\_1 -0.255  0.007      0.000    0.000    1002.0    1.0
\end{Verbatim}
\end{tcolorbox}
        
    \begin{tcolorbox}[breakable, size=fbox, boxrule=1pt, pad at break*=1mm,colback=cellbackground, colframe=cellborder]
\prompt{In}{incolor}{10}{\boxspacing}
\begin{Verbatim}[commandchars=\\\{\}]
\PY{n}{pm}\PY{o}{.}\PY{n}{traceplot}\PY{p}{(}\PY{n}{trace}\PY{p}{)}
\end{Verbatim}
\end{tcolorbox}

            \begin{tcolorbox}[breakable, size=fbox, boxrule=.5pt, pad at break*=1mm, opacityfill=0]
\prompt{Out}{outcolor}{10}{\boxspacing}
\begin{Verbatim}[commandchars=\\\{\}]
array([[<matplotlib.axes.\_subplots.AxesSubplot object at 0x7f7f2e6f7be0>,
        <matplotlib.axes.\_subplots.AxesSubplot object at 0x7f7f2e68e6a0>],
       [<matplotlib.axes.\_subplots.AxesSubplot object at 0x7f7f2e6bf9b0>,
        <matplotlib.axes.\_subplots.AxesSubplot object at 0x7f7f2e904cc0>]],
      dtype=object)
\end{Verbatim}
\end{tcolorbox}
        
    \begin{center}
    \adjustimage{max size={0.9\linewidth}{0.9\paperheight}}{output_22_1.png}
    \end{center}
    { \hspace*{\fill} \\}
    
    \begin{tcolorbox}[breakable, size=fbox, boxrule=1pt, pad at break*=1mm,colback=cellbackground, colframe=cellborder]
\prompt{In}{incolor}{11}{\boxspacing}
\begin{Verbatim}[commandchars=\\\{\}]
\PY{n}{pm}\PY{o}{.}\PY{n}{plot\PYZus{}posterior}\PY{p}{(}\PY{n}{trace}\PY{p}{)}
\end{Verbatim}
\end{tcolorbox}

            \begin{tcolorbox}[breakable, size=fbox, boxrule=.5pt, pad at break*=1mm, opacityfill=0]
\prompt{Out}{outcolor}{11}{\boxspacing}
\begin{Verbatim}[commandchars=\\\{\}]
array([<matplotlib.axes.\_subplots.AxesSubplot object at 0x7f7f2d47c630>,
       <matplotlib.axes.\_subplots.AxesSubplot object at 0x7f7f2d414588>],
      dtype=object)
\end{Verbatim}
\end{tcolorbox}
        
    \begin{center}
    \adjustimage{max size={0.9\linewidth}{0.9\paperheight}}{output_23_1.png}
    \end{center}
    { \hspace*{\fill} \\}
    
    From the above results, we can see:

\begin{itemize}
\tightlist
\item
  PyMC3 has estimated
\item
  \(b_0\) to be \(2.23 \pm 0.057\)
\item
  \(b_1\) to be \(-0.26 \pm 0.007\)
\item
  The MCSE is almost 0 \(\implies\) The simulation has run long enough
  for the chains to converge.
\item
  \(r\_hat = 1.0\) tells us that the chains have mixed well i.e hairy
  hedgehog pattern.
\end{itemize}

Let us plot the final output of this model and check it with our
training data.

    \begin{tcolorbox}[breakable, size=fbox, boxrule=1pt, pad at break*=1mm,colback=cellbackground, colframe=cellborder]
\prompt{In}{incolor}{12}{\boxspacing}
\begin{Verbatim}[commandchars=\\\{\}]
\PY{k}{with} \PY{n}{model}\PY{p}{:}
    \PY{n}{posterior\PYZus{}trace} \PY{o}{=} \PY{n}{pm}\PY{o}{.}\PY{n}{sample\PYZus{}posterior\PYZus{}predictive}\PY{p}{(}\PY{n}{trace}\PY{p}{)}
\end{Verbatim}
\end{tcolorbox}

    \begin{Verbatim}[commandchars=\\\{\}]
100\%|██████████| 4000/4000 [00:04<00:00, 987.81it/s]
    \end{Verbatim}

    \begin{tcolorbox}[breakable, size=fbox, boxrule=1pt, pad at break*=1mm,colback=cellbackground, colframe=cellborder]
\prompt{In}{incolor}{13}{\boxspacing}
\begin{Verbatim}[commandchars=\\\{\}]
\PY{n}{posterior\PYZus{}success} \PY{o}{=} \PY{n}{posterior\PYZus{}trace}\PY{p}{[}\PY{l+s+s1}{\PYZsq{}}\PY{l+s+s1}{y}\PY{l+s+s1}{\PYZsq{}}\PY{p}{]} \PY{o}{/} \PY{n}{df}\PY{o}{.}\PY{n}{tries}\PY{o}{.}\PY{n}{values}
\end{Verbatim}
\end{tcolorbox}

    \begin{tcolorbox}[breakable, size=fbox, boxrule=1pt, pad at break*=1mm,colback=cellbackground, colframe=cellborder]
\prompt{In}{incolor}{14}{\boxspacing}
\begin{Verbatim}[commandchars=\\\{\}]
\PY{n}{df}\PY{p}{[}\PY{l+s+s1}{\PYZsq{}}\PY{l+s+s1}{posterior\PYZus{}success\PYZus{}prob}\PY{l+s+s1}{\PYZsq{}}\PY{p}{]} \PY{o}{=} \PY{n}{pd}\PY{o}{.}\PY{n}{DataFrame}\PY{p}{(}\PY{n}{posterior\PYZus{}success}\PY{p}{)}\PY{o}{.}\PY{n}{median}\PY{p}{(}\PY{p}{)}
\PY{n}{df}\PY{p}{[}\PY{l+s+s1}{\PYZsq{}}\PY{l+s+s1}{posterior\PYZus{}success\PYZus{}prob\PYZus{}std}\PY{l+s+s1}{\PYZsq{}}\PY{p}{]} \PY{o}{=} \PY{n}{pd}\PY{o}{.}\PY{n}{DataFrame}\PY{p}{(}\PY{n}{posterior\PYZus{}success}\PY{p}{)}\PY{o}{.}\PY{n}{std}\PY{p}{(}\PY{p}{)}
\end{Verbatim}
\end{tcolorbox}

    \begin{tcolorbox}[breakable, size=fbox, boxrule=1pt, pad at break*=1mm,colback=cellbackground, colframe=cellborder]
\prompt{In}{incolor}{15}{\boxspacing}
\begin{Verbatim}[commandchars=\\\{\}]
\PY{n}{sns}\PY{o}{.}\PY{n}{set}\PY{p}{(}\PY{p}{)}
\PY{n}{plt}\PY{o}{.}\PY{n}{figure}\PY{p}{(}\PY{n}{figsize}\PY{o}{=}\PY{p}{(}\PY{l+m+mi}{16}\PY{p}{,} \PY{l+m+mi}{6}\PY{p}{)}\PY{p}{)}
\PY{n}{prob} \PY{o}{=} \PY{n}{df}\PY{o}{.}\PY{n}{success\PYZus{}count}\PY{o}{/}\PY{n}{df}\PY{o}{.}\PY{n}{tries}
\PY{n}{ax} \PY{o}{=} \PY{n}{sns}\PY{o}{.}\PY{n}{scatterplot}\PY{p}{(}\PY{n}{x}\PY{o}{=}\PY{l+s+s1}{\PYZsq{}}\PY{l+s+s1}{distance}\PY{l+s+s1}{\PYZsq{}}\PY{p}{,} \PY{n}{y}\PY{o}{=}\PY{n}{df}\PY{o}{.}\PY{n}{success\PYZus{}prob}\PY{p}{,} \PY{n}{data}\PY{o}{=}\PY{n}{df}\PY{p}{,} \PY{n}{s}\PY{o}{=}\PY{l+m+mi}{200}\PY{p}{,} \PY{n}{label}\PY{o}{=}\PY{l+s+s1}{\PYZsq{}}\PY{l+s+s1}{actual}\PY{l+s+s1}{\PYZsq{}}\PY{p}{)}
\PY{c+c1}{\PYZsh{} ls = np.linspace(0, df.distance.max(), 200)}
\PY{c+c1}{\PYZsh{} for index in np.random.randint(0, len(trace), 50):}
\PY{c+c1}{\PYZsh{}     ax.plot(}
\PY{c+c1}{\PYZsh{}         ls, }
\PY{c+c1}{\PYZsh{}         scipy.special.expit(}
\PY{c+c1}{\PYZsh{}             trace[\PYZsq{}b\PYZus{}0\PYZsq{}][index] * ls + trace[\PYZsq{}b\PYZus{}1\PYZsq{}][index] * ls}
\PY{c+c1}{\PYZsh{}         )}
\PY{c+c1}{\PYZsh{}     )}
\PY{n}{sns}\PY{o}{.}\PY{n}{scatterplot}\PY{p}{(}\PY{n}{x}\PY{o}{=}\PY{l+s+s1}{\PYZsq{}}\PY{l+s+s1}{distance}\PY{l+s+s1}{\PYZsq{}}\PY{p}{,} \PY{n}{y}\PY{o}{=}\PY{n}{df}\PY{o}{.}\PY{n}{posterior\PYZus{}success\PYZus{}prob}\PY{p}{,} \PY{n}{data}\PY{o}{=}\PY{n}{df}\PY{p}{,} \PY{n}{label}\PY{o}{=}\PY{l+s+s1}{\PYZsq{}}\PY{l+s+s1}{predicted}\PY{l+s+s1}{\PYZsq{}}\PY{p}{,}\PY{n}{ax}\PY{o}{=}\PY{n}{ax}\PY{p}{,} \PY{n}{color}\PY{o}{=}\PY{l+s+s1}{\PYZsq{}}\PY{l+s+s1}{red}\PY{l+s+s1}{\PYZsq{}}\PY{p}{,} \PY{n}{s}\PY{o}{=}\PY{l+m+mi}{200}\PY{p}{)}
\PY{n}{sns}\PY{o}{.}\PY{n}{lineplot}\PY{p}{(}\PY{n}{x}\PY{o}{=}\PY{l+s+s1}{\PYZsq{}}\PY{l+s+s1}{distance}\PY{l+s+s1}{\PYZsq{}}\PY{p}{,} \PY{n}{y}\PY{o}{=}\PY{n}{df}\PY{o}{.}\PY{n}{posterior\PYZus{}success\PYZus{}prob}\PY{p}{,} \PY{n}{data}\PY{o}{=}\PY{n}{df}\PY{p}{,}\PY{n}{ax}\PY{o}{=}\PY{n}{ax}\PY{p}{,} \PY{n}{color}\PY{o}{=}\PY{l+s+s1}{\PYZsq{}}\PY{l+s+s1}{red}\PY{l+s+s1}{\PYZsq{}}\PY{p}{)}
\PY{n}{ax}\PY{o}{.}\PY{n}{set}\PY{p}{(}\PY{n}{xlabel}\PY{o}{=}\PY{l+s+s1}{\PYZsq{}}\PY{l+s+s1}{Distance from hole(ft)}\PY{l+s+s1}{\PYZsq{}}\PY{p}{,} \PY{n}{ylabel}\PY{o}{=}\PY{l+s+s1}{\PYZsq{}}\PY{l+s+s1}{Probability of Success}\PY{l+s+s1}{\PYZsq{}}\PY{p}{)}
\end{Verbatim}
\end{tcolorbox}

            \begin{tcolorbox}[breakable, size=fbox, boxrule=.5pt, pad at break*=1mm, opacityfill=0]
\prompt{Out}{outcolor}{15}{\boxspacing}
\begin{Verbatim}[commandchars=\\\{\}]
[Text(0, 0.5, 'Probability of Success'),
 Text(0.5, 0, 'Distance from hole(ft)')]
\end{Verbatim}
\end{tcolorbox}
        
    \begin{center}
    \adjustimage{max size={0.9\linewidth}{0.9\paperheight}}{output_28_1.png}
    \end{center}
    { \hspace*{\fill} \\}
    
    The curve fit is okay, but it can be improved. We can use this as a
baseline model. In reality, each of them is not a point, but an
posterior estimate. Because the uncertainity is small(as seen above),
we've decided to show only the median points.

From the above model, putts from 50ft are expected to be made with
probability:

    \begin{tcolorbox}[breakable, size=fbox, boxrule=1pt, pad at break*=1mm,colback=cellbackground, colframe=cellborder]
\prompt{In}{incolor}{16}{\boxspacing}
\begin{Verbatim}[commandchars=\\\{\}]
\PY{k+kn}{import} \PY{n+nn}{scipy}
\PY{n}{res} \PY{o}{=} \PY{l+m+mi}{100} \PY{o}{*} \PY{n}{scipy}\PY{o}{.}\PY{n}{special}\PY{o}{.}\PY{n}{expit}\PY{p}{(}\PY{l+m+mf}{2.223} \PY{o}{+} \PY{o}{\PYZhy{}}\PY{l+m+mf}{0.255} \PY{o}{*} \PY{l+m+mi}{50}\PY{p}{)}\PY{o}{.}\PY{n}{mean}\PY{p}{(}\PY{p}{)}
\PY{n+nb}{print}\PY{p}{(}\PY{n}{np}\PY{o}{.}\PY{n}{round}\PY{p}{(}\PY{n}{res}\PY{p}{,} \PY{l+m+mi}{5}\PY{p}{)}\PY{p}{,}\PY{l+s+s2}{\PYZdq{}}\PY{l+s+s2}{\PYZpc{}}\PY{l+s+s2}{\PYZdq{}}\PY{p}{)}
\end{Verbatim}
\end{tcolorbox}

    \begin{Verbatim}[commandchars=\\\{\}]
0.00268 \%
    \end{Verbatim}

    \section{Modelling from first
principles}\label{modelling-from-first-principles}

    \subsection{Geometry based Model}\label{geometry-based-model}

    \begin{tcolorbox}[breakable, size=fbox, boxrule=1pt, pad at break*=1mm,colback=cellbackground, colframe=cellborder]
\prompt{In}{incolor}{17}{\boxspacing}
\begin{Verbatim}[commandchars=\\\{\}]
\PY{k+kn}{from} \PY{n+nn}{IPython}\PY{n+nn}{.}\PY{n+nn}{display} \PY{k+kn}{import} \PY{n}{Image}

\PY{n}{Image}\PY{p}{(}\PY{n}{url}\PY{o}{=}\PY{l+s+s1}{\PYZsq{}}\PY{l+s+s1}{./golf\PYZus{}ball\PYZus{}trajectory.png}\PY{l+s+s1}{\PYZsq{}}\PY{p}{)}
\end{Verbatim}
\end{tcolorbox}

            \begin{tcolorbox}[breakable, size=fbox, boxrule=.5pt, pad at break*=1mm, opacityfill=0]
\prompt{Out}{outcolor}{17}{\boxspacing}
\begin{Verbatim}[commandchars=\\\{\}]
<IPython.core.display.Image object>
\end{Verbatim}
\end{tcolorbox}
        
    We'll try to accomodate the physics associated with the problem.
Specically, we assume:

\subparagraph{Assumptions}\label{assumptions}

\begin{itemize}
\item
  The golfers can hit the ball in any direction with some small error.
  This error could be because of inaccuracy, errors in the human, etc.
\item
  This error refers to the angle of the shot.
\item
  We assume the angle is \textbf{normally} distributed.
\item
\end{itemize}

Implications

\begin{itemize}
\tightlist
\item
  The ball goes in whenever the angle is small enough for it to hit the
  cup of the hole!
\item
  Longer putt \(\implies\) Larger error \(\implies\) Lower success rate
  than shorter putt
\end{itemize}

From Dr. Gelman's blog, we obtain the formula as:

\begin{quote}
\(Pr(|angle| < sin^{-1}(\frac{(R-r)}{x})) = 2\phi\big(\frac{sin^{-1}\frac{R-r}{x}}{\sigma}\big) - 1\)
\end{quote}

\(\phi \implies\) Cumulative Normal Distribution Function.

    Hence, our model will now have two big parts:

\[y_j \sim binomial(n_j, p_j)\]

\[p_j = 2\phi\big(\frac{sin^{-1}\frac{R-r}{x}}{\sigma}\big) - 1\]

Typically, the diameter of a golf ball is 1.68 inches and the cup is
4.25 inches i.e

\[r = 1.68 \text{inch}\] \[R = 4.25 \text{inch}\]

    \begin{tcolorbox}[breakable, size=fbox, boxrule=1pt, pad at break*=1mm,colback=cellbackground, colframe=cellborder]
\prompt{In}{incolor}{18}{\boxspacing}
\begin{Verbatim}[commandchars=\\\{\}]
\PY{n}{ball\PYZus{}radius} \PY{o}{=} \PY{p}{(}\PY{l+m+mf}{1.68}\PY{o}{/}\PY{l+m+mi}{2}\PY{p}{)}\PY{o}{/}\PY{l+m+mi}{12}
\PY{n}{cup\PYZus{}radius} \PY{o}{=} \PY{p}{(}\PY{l+m+mf}{4.25}\PY{o}{/}\PY{l+m+mi}{2}\PY{p}{)}\PY{o}{/}\PY{l+m+mi}{12}
\end{Verbatim}
\end{tcolorbox}

    \begin{tcolorbox}[breakable, size=fbox, boxrule=1pt, pad at break*=1mm,colback=cellbackground, colframe=cellborder]
\prompt{In}{incolor}{33}{\boxspacing}
\begin{Verbatim}[commandchars=\\\{\}]
\PY{k}{def} \PY{n+nf}{calculate\PYZus{}prob}\PY{p}{(}\PY{n}{angle}\PY{p}{,} \PY{n}{distance}\PY{p}{)}\PY{p}{:}
    \PY{l+s+sd}{\PYZdq{}\PYZdq{}\PYZdq{}}
\PY{l+s+sd}{    \PYZdq{}\PYZdq{}\PYZdq{}}
    \PY{n}{rad} \PY{o}{=} \PY{n}{angle} \PY{o}{*} \PY{n}{np}\PY{o}{.}\PY{n}{pi} \PY{o}{/} \PY{l+m+mf}{180.0}
    \PY{n}{arcsin} \PY{o}{=} \PY{n}{np}\PY{o}{.}\PY{n}{arcsin}\PY{p}{(}\PY{p}{(}\PY{n}{cup\PYZus{}radius} \PY{o}{\PYZhy{}} \PY{n}{ball\PYZus{}radius}\PY{p}{)}\PY{o}{/} \PY{n}{distance}\PY{p}{)}
    \PY{k}{return} \PY{l+m+mi}{2} \PY{o}{*} \PY{n}{scipy}\PY{o}{.}\PY{n}{stats}\PY{o}{.}\PY{n}{norm}\PY{p}{(}\PY{l+m+mi}{0}\PY{p}{,} \PY{n}{rad}\PY{p}{)}\PY{o}{.}\PY{n}{cdf}\PY{p}{(}\PY{n}{arcsin}\PY{p}{)} \PY{o}{\PYZhy{}} \PY{l+m+mi}{1}
\end{Verbatim}
\end{tcolorbox}

    \begin{tcolorbox}[breakable, size=fbox, boxrule=1pt, pad at break*=1mm,colback=cellbackground, colframe=cellborder]
\prompt{In}{incolor}{34}{\boxspacing}
\begin{Verbatim}[commandchars=\\\{\}]
\PY{n}{plt}\PY{o}{.}\PY{n}{figure}\PY{p}{(}\PY{n}{figsize}\PY{o}{=}\PY{p}{(}\PY{l+m+mi}{16}\PY{p}{,} \PY{l+m+mi}{6}\PY{p}{)}\PY{p}{)}
\PY{n}{ls} \PY{o}{=} \PY{n}{np}\PY{o}{.}\PY{n}{linspace}\PY{p}{(}\PY{l+m+mi}{0}\PY{p}{,} \PY{n}{df}\PY{o}{.}\PY{n}{distance}\PY{o}{.}\PY{n}{max}\PY{p}{(}\PY{p}{)}\PY{p}{,} \PY{l+m+mi}{200}\PY{p}{)}
\PY{n}{ax} \PY{o}{=} \PY{n}{sns}\PY{o}{.}\PY{n}{scatterplot}\PY{p}{(}
    \PY{n}{x}\PY{o}{=}\PY{l+s+s1}{\PYZsq{}}\PY{l+s+s1}{distance}\PY{l+s+s1}{\PYZsq{}}\PY{p}{,} 
    \PY{n}{y}\PY{o}{=}\PY{l+s+s1}{\PYZsq{}}\PY{l+s+s1}{success\PYZus{}prob}\PY{l+s+s1}{\PYZsq{}}\PY{p}{,} 
    \PY{n}{data}\PY{o}{=}\PY{n}{df}\PY{p}{,} 
    \PY{n}{s}\PY{o}{=}\PY{l+m+mi}{100}\PY{p}{,}
    \PY{n}{legend}\PY{o}{=}\PY{l+s+s1}{\PYZsq{}}\PY{l+s+s1}{full}\PY{l+s+s1}{\PYZsq{}}
\PY{p}{)}
\PY{k}{for} \PY{n}{angle} \PY{o+ow}{in} \PY{p}{[}\PY{l+m+mf}{0.5}\PY{p}{,} \PY{l+m+mi}{1}\PY{p}{,} \PY{l+m+mi}{2}\PY{p}{,} \PY{l+m+mi}{5}\PY{p}{,} \PY{l+m+mi}{20}\PY{p}{]}\PY{p}{:}
    \PY{n}{ax}\PY{o}{.}\PY{n}{plot}\PY{p}{(}
        \PY{n}{ls}\PY{p}{,} 
        \PY{n}{calculate\PYZus{}prob}\PY{p}{(}\PY{n}{angle}\PY{p}{,} \PY{n}{ls}\PY{p}{)}\PY{p}{,} 
        \PY{n}{label}\PY{o}{=}\PY{l+s+sa}{f}\PY{l+s+s2}{\PYZdq{}}\PY{l+s+s2}{Angle=}\PY{l+s+si}{\PYZob{}angle\PYZcb{}}\PY{l+s+s2}{\PYZdq{}}
    \PY{p}{)}
\PY{n}{ax}\PY{o}{.}\PY{n}{set}\PY{p}{(}
    \PY{n}{xlabel}\PY{o}{=}\PY{l+s+s1}{\PYZsq{}}\PY{l+s+s1}{Distance from hole(ft)}\PY{l+s+s1}{\PYZsq{}}\PY{p}{,} 
    \PY{n}{ylabel}\PY{o}{=}\PY{l+s+s1}{\PYZsq{}}\PY{l+s+s1}{Probability of Success}\PY{l+s+s1}{\PYZsq{}}
\PY{p}{)}
\PY{n}{ax}\PY{o}{.}\PY{n}{legend}\PY{p}{(}\PY{p}{)}
\end{Verbatim}
\end{tcolorbox}

            \begin{tcolorbox}[breakable, size=fbox, boxrule=.5pt, pad at break*=1mm, opacityfill=0]
\prompt{Out}{outcolor}{34}{\boxspacing}
\begin{Verbatim}[commandchars=\\\{\}]
<matplotlib.legend.Legend at 0x7f7f2e7d50f0>
\end{Verbatim}
\end{tcolorbox}
        
    \begin{center}
    \adjustimage{max size={0.9\linewidth}{0.9\paperheight}}{output_38_1.png}
    \end{center}
    { \hspace*{\fill} \\}
    
    Let us now add this to our model!

    \begin{tcolorbox}[breakable, size=fbox, boxrule=1pt, pad at break*=1mm,colback=cellbackground, colframe=cellborder]
\prompt{In}{incolor}{21}{\boxspacing}
\begin{Verbatim}[commandchars=\\\{\}]
\PY{k+kn}{import} \PY{n+nn}{theano}\PY{n+nn}{.}\PY{n+nn}{tensor} \PY{k}{as} \PY{n+nn}{tt}


\PY{k}{def} \PY{n+nf}{calculate\PYZus{}phi}\PY{p}{(}\PY{n}{num}\PY{p}{)}\PY{p}{:}
    \PY{l+s+s2}{\PYZdq{}}\PY{l+s+s2}{cdf for standard normal}\PY{l+s+s2}{\PYZdq{}}
    \PY{n}{q} \PY{o}{=} \PY{n}{tt}\PY{o}{.}\PY{n}{erf}\PY{p}{(}\PY{n}{num} \PY{o}{/} \PY{n}{tt}\PY{o}{.}\PY{n}{sqrt}\PY{p}{(}\PY{l+m+mf}{2.0}\PY{p}{)}\PY{p}{)} \PY{c+c1}{\PYZsh{} ERF is the Gaussian Error }
    \PY{k}{return} \PY{p}{(}\PY{l+m+mf}{1.0} \PY{o}{+} \PY{n}{q}\PY{p}{)} \PY{o}{/} \PY{l+m+mf}{2.}
\end{Verbatim}
\end{tcolorbox}

    \begin{tcolorbox}[breakable, size=fbox, boxrule=1pt, pad at break*=1mm,colback=cellbackground, colframe=cellborder]
\prompt{In}{incolor}{53}{\boxspacing}
\begin{Verbatim}[commandchars=\\\{\}]
\PY{k}{with} \PY{n}{pm}\PY{o}{.}\PY{n}{Model}\PY{p}{(}\PY{p}{)} \PY{k}{as} \PY{n}{model}\PY{p}{:}
    \PY{n}{angle\PYZus{}of\PYZus{}shot\PYZus{}radians} \PY{o}{=} \PY{n}{pm}\PY{o}{.}\PY{n}{HalfNormal}\PY{p}{(}\PY{l+s+s1}{\PYZsq{}}\PY{l+s+s1}{angle\PYZus{}of\PYZus{}shot\PYZus{}radians}\PY{l+s+s1}{\PYZsq{}}\PY{p}{)}
    \PY{n}{angle\PYZus{}of\PYZus{}shot\PYZus{}degrees} \PY{o}{=} \PY{n}{pm}\PY{o}{.}\PY{n}{Deterministic}\PY{p}{(}
        \PY{l+s+s1}{\PYZsq{}}\PY{l+s+s1}{angle\PYZus{}of\PYZus{}shot\PYZus{}degrees}\PY{l+s+s1}{\PYZsq{}}\PY{p}{,}
        \PY{p}{(}\PY{n}{angle\PYZus{}of\PYZus{}shot\PYZus{}radians} \PY{o}{*} \PY{l+m+mf}{180.0}\PY{p}{)} \PY{o}{/} \PY{n}{np}\PY{o}{.}\PY{n}{pi}
    \PY{p}{)}
    \PY{n}{p\PYZus{}ball\PYZus{}goes\PYZus{}in} \PY{o}{=} \PY{n}{pm}\PY{o}{.}\PY{n}{Deterministic}\PY{p}{(}
        \PY{l+s+s1}{\PYZsq{}}\PY{l+s+s1}{p\PYZus{}ball\PYZus{}goes\PYZus{}in}\PY{l+s+s1}{\PYZsq{}}\PY{p}{,}
        \PY{l+m+mi}{2} \PY{o}{*} \PY{n}{calculate\PYZus{}phi}\PY{p}{(}
                \PY{n}{tt}\PY{o}{.}\PY{n}{arcsin}\PY{p}{(}
                    \PY{p}{(}\PY{n}{cup\PYZus{}radius} \PY{o}{\PYZhy{}} \PY{n}{ball\PYZus{}radius}\PY{p}{)}\PY{o}{/} \PY{n}{df}\PY{o}{.}\PY{n}{distance}
                \PY{p}{)} \PY{o}{/} \PY{n}{angle\PYZus{}of\PYZus{}shot\PYZus{}radians}
            \PY{p}{)}
        \PY{p}{)} \PY{o}{\PYZhy{}} \PY{l+m+mi}{1}
    \PY{n}{p\PYZus{}success} \PY{o}{=} \PY{n}{pm}\PY{o}{.}\PY{n}{Binomial}\PY{p}{(}
        \PY{l+s+s1}{\PYZsq{}}\PY{l+s+s1}{p\PYZus{}success}\PY{l+s+s1}{\PYZsq{}}\PY{p}{,}
        \PY{n}{n}\PY{o}{=}\PY{n}{df}\PY{o}{.}\PY{n}{tries}\PY{p}{,} 
        \PY{n}{p}\PY{o}{=}\PY{n}{p\PYZus{}ball\PYZus{}goes\PYZus{}in}\PY{p}{,} 
        \PY{n}{observed}\PY{o}{=}\PY{n}{df}\PY{o}{.}\PY{n}{success\PYZus{}count}
    \PY{p}{)}
\end{Verbatim}
\end{tcolorbox}

    \begin{tcolorbox}[breakable, size=fbox, boxrule=1pt, pad at break*=1mm,colback=cellbackground, colframe=cellborder]
\prompt{In}{incolor}{54}{\boxspacing}
\begin{Verbatim}[commandchars=\\\{\}]
\PY{n}{pm}\PY{o}{.}\PY{n}{model\PYZus{}to\PYZus{}graphviz}\PY{p}{(}\PY{n}{model}\PY{p}{)}
\end{Verbatim}
\end{tcolorbox}
 
            
\prompt{Out}{outcolor}{54}{}
    
    \begin{center}
    \adjustimage{max size={0.9\linewidth}{0.9\paperheight}}{output_42_0.pdf}
    \end{center}
    { \hspace*{\fill} \\}
    

    \begin{tcolorbox}[breakable, size=fbox, boxrule=1pt, pad at break*=1mm,colback=cellbackground, colframe=cellborder]
\prompt{In}{incolor}{55}{\boxspacing}
\begin{Verbatim}[commandchars=\\\{\}]
\PY{k}{with} \PY{n}{model}\PY{p}{:}
    \PY{n}{trace} \PY{o}{=} \PY{n}{pm}\PY{o}{.}\PY{n}{sample}\PY{p}{(}\PY{l+m+mi}{4000}\PY{p}{,} \PY{n}{tune}\PY{o}{=}\PY{l+m+mi}{1000}\PY{p}{,} \PY{n}{chains}\PY{o}{=}\PY{l+m+mi}{4}\PY{p}{)}
\end{Verbatim}
\end{tcolorbox}

    \begin{Verbatim}[commandchars=\\\{\}]
Auto-assigning NUTS sampler{\ldots}
Initializing NUTS using jitter+adapt\_diag{\ldots}
ERROR (theano.gof.opt): Optimization failure due to: local\_grad\_log\_erfc\_neg
ERROR (theano.gof.opt): node: Elemwise\{true\_div\}(Elemwise\{mul,no\_inplace\}.0,
Elemwise\{erfc,no\_inplace\}.0)
ERROR (theano.gof.opt): TRACEBACK:
ERROR (theano.gof.opt): Traceback (most recent call last):
  File "/home/goodhamgupta/shubham/golf\_tutorial/.env/lib/python3.6/site-
packages/theano/gof/opt.py", line 2034, in process\_node
    replacements = lopt.transform(node)
  File "/home/goodhamgupta/shubham/golf\_tutorial/.env/lib/python3.6/site-
packages/theano/tensor/opt.py", line 6789, in local\_grad\_log\_erfc\_neg
    if not exp.owner.inputs[0].owner:
AttributeError: 'NoneType' object has no attribute 'owner'

Multiprocess sampling (4 chains in 2 jobs)
NUTS: [angle\_of\_shot\_radians]
Sampling 4 chains, 0 divergences: 100\%|██████████| 20000/20000 [00:09<00:00,
2046.72draws/s]
    \end{Verbatim}

    \begin{tcolorbox}[breakable, size=fbox, boxrule=1pt, pad at break*=1mm,colback=cellbackground, colframe=cellborder]
\prompt{In}{incolor}{65}{\boxspacing}
\begin{Verbatim}[commandchars=\\\{\}]
\PY{n}{pm}\PY{o}{.}\PY{n}{summary}\PY{p}{(}\PY{n}{trace}\PY{p}{)}\PY{o}{.}\PY{n}{head}\PY{p}{(}\PY{l+m+mi}{2}\PY{p}{)}
\end{Verbatim}
\end{tcolorbox}

            \begin{tcolorbox}[breakable, size=fbox, boxrule=.5pt, pad at break*=1mm, opacityfill=0]
\prompt{Out}{outcolor}{65}{\boxspacing}
\begin{Verbatim}[commandchars=\\\{\}]
                        mean     sd  hpd\_3\%  hpd\_97\%  mcse\_mean  mcse\_sd  \textbackslash{}
angle\_of\_shot\_radians  0.027  0.000   0.026    0.027        0.0      0.0
angle\_of\_shot\_degrees  1.528  0.023   1.485    1.570        0.0      0.0

                       ess\_mean  ess\_sd  ess\_bulk  ess\_tail  r\_hat
angle\_of\_shot\_radians    6649.0  6649.0    6658.0   12054.0    1.0
angle\_of\_shot\_degrees    6649.0  6649.0    6658.0   12054.0    1.0
\end{Verbatim}
\end{tcolorbox}
        
    \begin{tcolorbox}[breakable, size=fbox, boxrule=1pt, pad at break*=1mm,colback=cellbackground, colframe=cellborder]
\prompt{In}{incolor}{66}{\boxspacing}
\begin{Verbatim}[commandchars=\\\{\}]
\PY{n}{pm}\PY{o}{.}\PY{n}{plot\PYZus{}posterior}\PY{p}{(}\PY{n}{trace}\PY{p}{[}\PY{l+s+s1}{\PYZsq{}}\PY{l+s+s1}{angle\PYZus{}of\PYZus{}shot\PYZus{}degrees}\PY{l+s+s1}{\PYZsq{}}\PY{p}{]}\PY{p}{)}
\end{Verbatim}
\end{tcolorbox}

            \begin{tcolorbox}[breakable, size=fbox, boxrule=.5pt, pad at break*=1mm, opacityfill=0]
\prompt{Out}{outcolor}{66}{\boxspacing}
\begin{Verbatim}[commandchars=\\\{\}]
array([<matplotlib.axes.\_subplots.AxesSubplot object at 0x7f7f184de8d0>],
      dtype=object)
\end{Verbatim}
\end{tcolorbox}
        
    \begin{center}
    \adjustimage{max size={0.9\linewidth}{0.9\paperheight}}{output_45_1.png}
    \end{center}
    { \hspace*{\fill} \\}
    
    From the above results, we can see:

\begin{itemize}
\tightlist
\item
  PyMC3 has estimated
\item
  \(angle_of_shot_degrees\) to be \(1.53 \pm 0.023\)
\item
  The MCSE is almost 0 \(\implies\) The simulation has run long enough
  for the chains to converge.
\item
  \(r\_hat = 1.0\) tells us that the chains have mixed well i.e hairy
  hedgehog pattern.
\end{itemize}

    Let's visualize the fit with this new model:

    \begin{tcolorbox}[breakable, size=fbox, boxrule=1pt, pad at break*=1mm,colback=cellbackground, colframe=cellborder]
\prompt{In}{incolor}{67}{\boxspacing}
\begin{Verbatim}[commandchars=\\\{\}]
\PY{n}{geo\PYZus{}model\PYZus{}prob} \PY{o}{=} \PY{n}{calculate\PYZus{}prob}\PY{p}{(}
    \PY{n}{trace}\PY{p}{[}\PY{l+s+s1}{\PYZsq{}}\PY{l+s+s1}{angle\PYZus{}of\PYZus{}shot\PYZus{}degrees}\PY{l+s+s1}{\PYZsq{}}\PY{p}{]}\PY{o}{.}\PY{n}{mean}\PY{p}{(}\PY{p}{)}\PY{p}{,} 
    \PY{n}{df}\PY{o}{.}\PY{n}{distance}
\PY{p}{)}
\end{Verbatim}
\end{tcolorbox}

    \begin{tcolorbox}[breakable, size=fbox, boxrule=1pt, pad at break*=1mm,colback=cellbackground, colframe=cellborder]
\prompt{In}{incolor}{68}{\boxspacing}
\begin{Verbatim}[commandchars=\\\{\}]
\PY{n}{sns}\PY{o}{.}\PY{n}{set}\PY{p}{(}\PY{p}{)}
\PY{n}{plt}\PY{o}{.}\PY{n}{figure}\PY{p}{(}\PY{n}{figsize}\PY{o}{=}\PY{p}{(}\PY{l+m+mi}{16}\PY{p}{,} \PY{l+m+mi}{6}\PY{p}{)}\PY{p}{)}

\PY{n}{ax} \PY{o}{=} \PY{n}{sns}\PY{o}{.}\PY{n}{scatterplot}\PY{p}{(}\PY{n}{x}\PY{o}{=}\PY{l+s+s1}{\PYZsq{}}\PY{l+s+s1}{distance}\PY{l+s+s1}{\PYZsq{}}\PY{p}{,} \PY{n}{y}\PY{o}{=}\PY{n}{df}\PY{o}{.}\PY{n}{success\PYZus{}prob}\PY{p}{,} \PY{n}{data}\PY{o}{=}\PY{n}{df}\PY{p}{,} \PY{n}{s}\PY{o}{=}\PY{l+m+mi}{200}\PY{p}{,} \PY{n}{label}\PY{o}{=}\PY{l+s+s1}{\PYZsq{}}\PY{l+s+s1}{Actual}\PY{l+s+s1}{\PYZsq{}}\PY{p}{)}

\PY{n}{sns}\PY{o}{.}\PY{n}{scatterplot}\PY{p}{(}\PY{n}{x}\PY{o}{=}\PY{l+s+s1}{\PYZsq{}}\PY{l+s+s1}{distance}\PY{l+s+s1}{\PYZsq{}}\PY{p}{,} \PY{n}{y}\PY{o}{=}\PY{n}{df}\PY{o}{.}\PY{n}{posterior\PYZus{}success\PYZus{}prob}\PY{p}{,} \PY{n}{data}\PY{o}{=}\PY{n}{df}\PY{p}{,} \PY{n}{label}\PY{o}{=}\PY{l+s+s1}{\PYZsq{}}\PY{l+s+s1}{Logistic Regression}\PY{l+s+s1}{\PYZsq{}}\PY{p}{,}\PY{n}{ax}\PY{o}{=}\PY{n}{ax}\PY{p}{,} \PY{n}{color}\PY{o}{=}\PY{l+s+s1}{\PYZsq{}}\PY{l+s+s1}{red}\PY{l+s+s1}{\PYZsq{}}\PY{p}{,} \PY{n}{s}\PY{o}{=}\PY{l+m+mi}{100}\PY{p}{)}
\PY{n}{sns}\PY{o}{.}\PY{n}{scatterplot}\PY{p}{(}\PY{n}{x}\PY{o}{=}\PY{l+s+s1}{\PYZsq{}}\PY{l+s+s1}{distance}\PY{l+s+s1}{\PYZsq{}}\PY{p}{,} \PY{n}{y}\PY{o}{=}\PY{n}{geo\PYZus{}model\PYZus{}prob}\PY{p}{,} \PY{n}{data}\PY{o}{=}\PY{n}{df}\PY{p}{,} \PY{n}{label}\PY{o}{=}\PY{l+s+s1}{\PYZsq{}}\PY{l+s+s1}{Geometry based }\PY{l+s+s1}{\PYZsq{}}\PY{p}{,}\PY{n}{ax}\PY{o}{=}\PY{n}{ax}\PY{p}{,} \PY{n}{color}\PY{o}{=}\PY{l+s+s1}{\PYZsq{}}\PY{l+s+s1}{orange}\PY{l+s+s1}{\PYZsq{}}\PY{p}{,} \PY{n}{s}\PY{o}{=}\PY{l+m+mi}{100}\PY{p}{)}

\PY{n}{sns}\PY{o}{.}\PY{n}{lineplot}\PY{p}{(}\PY{n}{x}\PY{o}{=}\PY{l+s+s1}{\PYZsq{}}\PY{l+s+s1}{distance}\PY{l+s+s1}{\PYZsq{}}\PY{p}{,} \PY{n}{y}\PY{o}{=}\PY{n}{df}\PY{o}{.}\PY{n}{posterior\PYZus{}success\PYZus{}prob}\PY{p}{,} \PY{n}{data}\PY{o}{=}\PY{n}{df}\PY{p}{,}\PY{n}{ax}\PY{o}{=}\PY{n}{ax}\PY{p}{,} \PY{n}{color}\PY{o}{=}\PY{l+s+s1}{\PYZsq{}}\PY{l+s+s1}{red}\PY{l+s+s1}{\PYZsq{}}\PY{p}{)}
\PY{n}{sns}\PY{o}{.}\PY{n}{lineplot}\PY{p}{(}\PY{n}{x}\PY{o}{=}\PY{l+s+s1}{\PYZsq{}}\PY{l+s+s1}{distance}\PY{l+s+s1}{\PYZsq{}}\PY{p}{,} \PY{n}{y}\PY{o}{=}\PY{n}{geo\PYZus{}model\PYZus{}prob}\PY{p}{,} \PY{n}{data}\PY{o}{=}\PY{n}{df}\PY{p}{,}\PY{n}{ax}\PY{o}{=}\PY{n}{ax}\PY{p}{,} \PY{n}{color}\PY{o}{=}\PY{l+s+s1}{\PYZsq{}}\PY{l+s+s1}{orange}\PY{l+s+s1}{\PYZsq{}}\PY{p}{)}

\PY{n}{ax}\PY{o}{.}\PY{n}{set}\PY{p}{(}\PY{n}{xlabel}\PY{o}{=}\PY{l+s+s1}{\PYZsq{}}\PY{l+s+s1}{Distance from hole(ft)}\PY{l+s+s1}{\PYZsq{}}\PY{p}{,} \PY{n}{ylabel}\PY{o}{=}\PY{l+s+s1}{\PYZsq{}}\PY{l+s+s1}{Probability of Success}\PY{l+s+s1}{\PYZsq{}}\PY{p}{)}
\end{Verbatim}
\end{tcolorbox}

            \begin{tcolorbox}[breakable, size=fbox, boxrule=.5pt, pad at break*=1mm, opacityfill=0]
\prompt{Out}{outcolor}{68}{\boxspacing}
\begin{Verbatim}[commandchars=\\\{\}]
[Text(0, 0.5, 'Probability of Success'),
 Text(0.5, 0, 'Distance from hole(ft)')]
\end{Verbatim}
\end{tcolorbox}
        
    \begin{center}
    \adjustimage{max size={0.9\linewidth}{0.9\paperheight}}{output_49_1.png}
    \end{center}
    { \hspace*{\fill} \\}
    
    \begin{itemize}
\tightlist
\item
  We can see that the geometry based model fits better than the logistic
  regression model.
\item
  While this model is not completely accurate, it suggests that angle is
  a good variable to model the problem. Using this model, we can be more
  confident about extrapolating the data.
\item
  For the same 50ft putt, the probability now is:
\end{itemize}

    \begin{tcolorbox}[breakable, size=fbox, boxrule=1pt, pad at break*=1mm,colback=cellbackground, colframe=cellborder]
\prompt{In}{incolor}{30}{\boxspacing}
\begin{Verbatim}[commandchars=\\\{\}]
\PY{k+kn}{import} \PY{n+nn}{scipy}
\PY{n}{lr\PYZus{}result} \PY{o}{=} \PY{n}{np}\PY{o}{.}\PY{n}{round}\PY{p}{(}
    \PY{l+m+mi}{100} \PY{o}{*} \PY{n}{scipy}\PY{o}{.}\PY{n}{special}\PY{o}{.}\PY{n}{expit}\PY{p}{(}\PY{l+m+mf}{2.223} \PY{o}{+} \PY{o}{\PYZhy{}}\PY{l+m+mf}{0.255} \PY{o}{*} \PY{l+m+mi}{50}\PY{p}{)}\PY{o}{.}\PY{n}{mean}\PY{p}{(}\PY{p}{)}\PY{p}{,}
    \PY{l+m+mi}{5}
\PY{p}{)}
\PY{n}{geo\PYZus{}result} \PY{o}{=} \PY{n}{np}\PY{o}{.}\PY{n}{round}\PY{p}{(}
    \PY{l+m+mi}{100} \PY{o}{*} \PY{n}{calculate\PYZus{}prob}\PY{p}{(}
        \PY{n}{trace}\PY{p}{[}\PY{l+s+s1}{\PYZsq{}}\PY{l+s+s1}{angle\PYZus{}of\PYZus{}shot\PYZus{}degrees}\PY{l+s+s1}{\PYZsq{}}\PY{p}{]}\PY{o}{.}\PY{n}{mean}\PY{p}{(}\PY{p}{)}\PY{p}{,} 
        \PY{l+m+mi}{50}
    \PY{p}{)}\PY{o}{.}\PY{n}{mean}\PY{p}{(}\PY{p}{)}\PY{p}{,}
    \PY{l+m+mi}{5}
\PY{p}{)}

\PY{n+nb}{print}\PY{p}{(}
    \PY{l+s+sa}{f}\PY{l+s+s2}{\PYZdq{}}\PY{l+s+s2}{Logistic Regression Model: }\PY{l+s+si}{\PYZob{}lr\PYZus{}result\PYZcb{}}\PY{l+s+s2}{\PYZpc{}}\PY{l+s+s2}{ }\PY{l+s+se}{\PYZbs{}n}\PY{l+s+s2}{\PYZdq{}}\PYZbs{}
    \PY{l+s+sa}{f}\PY{l+s+s2}{\PYZdq{}}\PY{l+s+s2}{Geometry Based Model: }\PY{l+s+si}{\PYZob{}geo\PYZus{}result\PYZcb{}}\PY{l+s+s2}{\PYZpc{}}\PY{l+s+s2}{\PYZdq{}}
\PY{p}{)}
\end{Verbatim}
\end{tcolorbox}

    \begin{Verbatim}[commandchars=\\\{\}]
Logistic Regression Model: 0.00268\%
Geometry Based Model: 6.4021\%
    \end{Verbatim}

    \subsection{New Data!}\label{new-data}

    Mark Broadie obtained new data about the golfers. Let's see how our
model performs on this new dataset.

First, we'll look at the summary of the dataset.

    \begin{tcolorbox}[breakable, size=fbox, boxrule=1pt, pad at break*=1mm,colback=cellbackground, colframe=cellborder]
\prompt{In}{incolor}{77}{\boxspacing}
\begin{Verbatim}[commandchars=\\\{\}]
\PY{c+c1}{\PYZsh{}  golf putting data from Broadie (2018)}
\PY{n}{new\PYZus{}golf\PYZus{}data} \PY{o}{=} \PY{n}{np}\PY{o}{.}\PY{n}{array}\PY{p}{(}\PY{p}{[}
\PY{p}{[}\PY{l+m+mf}{0.28}\PY{p}{,} \PY{l+m+mi}{45198}\PY{p}{,} \PY{l+m+mi}{45183}\PY{p}{]}\PY{p}{,}
\PY{p}{[}\PY{l+m+mf}{0.97}\PY{p}{,} \PY{l+m+mi}{183020}\PY{p}{,} \PY{l+m+mi}{182899}\PY{p}{]}\PY{p}{,}
\PY{p}{[}\PY{l+m+mf}{1.93}\PY{p}{,} \PY{l+m+mi}{169503}\PY{p}{,} \PY{l+m+mi}{168594}\PY{p}{]}\PY{p}{,}
\PY{p}{[}\PY{l+m+mf}{2.92}\PY{p}{,} \PY{l+m+mi}{113094}\PY{p}{,} \PY{l+m+mi}{108953}\PY{p}{]}\PY{p}{,}
\PY{p}{[}\PY{l+m+mf}{3.93}\PY{p}{,} \PY{l+m+mi}{73855}\PY{p}{,} \PY{l+m+mi}{64740}\PY{p}{]}\PY{p}{,}
\PY{p}{[}\PY{l+m+mf}{4.94}\PY{p}{,} \PY{l+m+mi}{53659}\PY{p}{,} \PY{l+m+mi}{41106}\PY{p}{]}\PY{p}{,}
\PY{p}{[}\PY{l+m+mf}{5.94}\PY{p}{,} \PY{l+m+mi}{42991}\PY{p}{,} \PY{l+m+mi}{28205}\PY{p}{]}\PY{p}{,}
\PY{p}{[}\PY{l+m+mf}{6.95}\PY{p}{,} \PY{l+m+mi}{37050}\PY{p}{,} \PY{l+m+mi}{21334}\PY{p}{]}\PY{p}{,}
\PY{p}{[}\PY{l+m+mf}{7.95}\PY{p}{,} \PY{l+m+mi}{33275}\PY{p}{,} \PY{l+m+mi}{16615}\PY{p}{]}\PY{p}{,}
\PY{p}{[}\PY{l+m+mf}{8.95}\PY{p}{,} \PY{l+m+mi}{30836}\PY{p}{,} \PY{l+m+mi}{13503}\PY{p}{]}\PY{p}{,}
\PY{p}{[}\PY{l+m+mf}{9.95}\PY{p}{,} \PY{l+m+mi}{28637}\PY{p}{,} \PY{l+m+mi}{11060}\PY{p}{]}\PY{p}{,}
\PY{p}{[}\PY{l+m+mf}{10.95}\PY{p}{,} \PY{l+m+mi}{26239}\PY{p}{,} \PY{l+m+mi}{9032}\PY{p}{]}\PY{p}{,}
\PY{p}{[}\PY{l+m+mf}{11.95}\PY{p}{,} \PY{l+m+mi}{24636}\PY{p}{,} \PY{l+m+mi}{7687}\PY{p}{]}\PY{p}{,}
\PY{p}{[}\PY{l+m+mf}{12.95}\PY{p}{,} \PY{l+m+mi}{22876}\PY{p}{,} \PY{l+m+mi}{6432}\PY{p}{]}\PY{p}{,}
\PY{p}{[}\PY{l+m+mf}{14.43}\PY{p}{,} \PY{l+m+mi}{41267}\PY{p}{,} \PY{l+m+mi}{9813}\PY{p}{]}\PY{p}{,}
\PY{p}{[}\PY{l+m+mf}{16.43}\PY{p}{,} \PY{l+m+mi}{35712}\PY{p}{,} \PY{l+m+mi}{7196}\PY{p}{]}\PY{p}{,}
\PY{p}{[}\PY{l+m+mf}{18.44}\PY{p}{,} \PY{l+m+mi}{31573}\PY{p}{,} \PY{l+m+mi}{5290}\PY{p}{]}\PY{p}{,}
\PY{p}{[}\PY{l+m+mf}{20.44}\PY{p}{,} \PY{l+m+mi}{28280}\PY{p}{,} \PY{l+m+mi}{4086}\PY{p}{]}\PY{p}{,}
\PY{p}{[}\PY{l+m+mf}{21.95}\PY{p}{,} \PY{l+m+mi}{13238}\PY{p}{,} \PY{l+m+mi}{1642}\PY{p}{]}\PY{p}{,}
\PY{p}{[}\PY{l+m+mf}{24.39}\PY{p}{,} \PY{l+m+mi}{46570}\PY{p}{,} \PY{l+m+mi}{4767}\PY{p}{]}\PY{p}{,}
\PY{p}{[}\PY{l+m+mf}{28.40}\PY{p}{,} \PY{l+m+mi}{38422}\PY{p}{,} \PY{l+m+mi}{2980}\PY{p}{]}\PY{p}{,}
\PY{p}{[}\PY{l+m+mf}{32.39}\PY{p}{,} \PY{l+m+mi}{31641}\PY{p}{,} \PY{l+m+mi}{1996}\PY{p}{]}\PY{p}{,}
\PY{p}{[}\PY{l+m+mf}{36.39}\PY{p}{,} \PY{l+m+mi}{25604}\PY{p}{,} \PY{l+m+mi}{1327}\PY{p}{]}\PY{p}{,}
\PY{p}{[}\PY{l+m+mf}{40.37}\PY{p}{,} \PY{l+m+mi}{20366}\PY{p}{,} \PY{l+m+mi}{834}\PY{p}{]}\PY{p}{,}
\PY{p}{[}\PY{l+m+mf}{44.38}\PY{p}{,} \PY{l+m+mi}{15977}\PY{p}{,} \PY{l+m+mi}{559}\PY{p}{]}\PY{p}{,}
\PY{p}{[}\PY{l+m+mf}{48.37}\PY{p}{,} \PY{l+m+mi}{11770}\PY{p}{,} \PY{l+m+mi}{311}\PY{p}{]}\PY{p}{,}
\PY{p}{[}\PY{l+m+mf}{52.36}\PY{p}{,} \PY{l+m+mi}{8708}\PY{p}{,} \PY{l+m+mi}{231}\PY{p}{]}\PY{p}{,}
\PY{p}{[}\PY{l+m+mf}{57.25}\PY{p}{,} \PY{l+m+mi}{8878}\PY{p}{,} \PY{l+m+mi}{204}\PY{p}{]}\PY{p}{,}
\PY{p}{[}\PY{l+m+mf}{63.23}\PY{p}{,} \PY{l+m+mi}{5492}\PY{p}{,} \PY{l+m+mi}{103}\PY{p}{]}\PY{p}{,}
\PY{p}{[}\PY{l+m+mf}{69.18}\PY{p}{,} \PY{l+m+mi}{3087}\PY{p}{,} \PY{l+m+mi}{35}\PY{p}{]}\PY{p}{,}
\PY{p}{[}\PY{l+m+mf}{75.19}\PY{p}{,} \PY{l+m+mi}{1742}\PY{p}{,} \PY{l+m+mi}{24}\PY{p}{]}\PY{p}{,}
\PY{p}{]}\PY{p}{)}

\PY{n}{new\PYZus{}df} \PY{o}{=} \PY{n}{pd}\PY{o}{.}\PY{n}{DataFrame}\PY{p}{(}
    \PY{n}{new\PYZus{}golf\PYZus{}data}\PY{p}{,} 
    \PY{n}{columns}\PY{o}{=}\PY{p}{[}\PY{l+s+s1}{\PYZsq{}}\PY{l+s+s1}{distance}\PY{l+s+s1}{\PYZsq{}}\PY{p}{,} \PY{l+s+s1}{\PYZsq{}}\PY{l+s+s1}{tries}\PY{l+s+s1}{\PYZsq{}}\PY{p}{,} \PY{l+s+s1}{\PYZsq{}}\PY{l+s+s1}{success\PYZus{}count}\PY{l+s+s1}{\PYZsq{}}\PY{p}{]}
\PY{p}{)}
\end{Verbatim}
\end{tcolorbox}

    \begin{tcolorbox}[breakable, size=fbox, boxrule=1pt, pad at break*=1mm,colback=cellbackground, colframe=cellborder]
\prompt{In}{incolor}{84}{\boxspacing}
\begin{Verbatim}[commandchars=\\\{\}]
\PY{n}{new\PYZus{}geo\PYZus{}model\PYZus{}prob} \PY{o}{=} \PY{n}{calculate\PYZus{}prob}\PY{p}{(}
    \PY{n}{trace}\PY{p}{[}\PY{l+s+s1}{\PYZsq{}}\PY{l+s+s1}{angle\PYZus{}of\PYZus{}shot\PYZus{}degrees}\PY{l+s+s1}{\PYZsq{}}\PY{p}{]}\PY{o}{.}\PY{n}{mean}\PY{p}{(}\PY{p}{)}\PY{p}{,} 
    \PY{n}{new\PYZus{}df}\PY{o}{.}\PY{n}{distance}
\PY{p}{)}
\end{Verbatim}
\end{tcolorbox}

    \begin{tcolorbox}[breakable, size=fbox, boxrule=1pt, pad at break*=1mm,colback=cellbackground, colframe=cellborder]
\prompt{In}{incolor}{94}{\boxspacing}
\begin{Verbatim}[commandchars=\\\{\}]
\PY{n}{new\PYZus{}df}\PY{p}{[}\PY{l+s+s1}{\PYZsq{}}\PY{l+s+s1}{success\PYZus{}prob}\PY{l+s+s1}{\PYZsq{}}\PY{p}{]} \PY{o}{=} \PY{n}{new\PYZus{}df}\PY{o}{.}\PY{n}{success\PYZus{}count} \PY{o}{/} \PY{n}{new\PYZus{}df}\PY{o}{.}\PY{n}{tries}
\PY{n}{sns}\PY{o}{.}\PY{n}{set}\PY{p}{(}\PY{p}{)}
\PY{n}{plt}\PY{o}{.}\PY{n}{figure}\PY{p}{(}\PY{n}{figsize}\PY{o}{=}\PY{p}{(}\PY{l+m+mi}{16}\PY{p}{,} \PY{l+m+mi}{6}\PY{p}{)}\PY{p}{)}
\PY{n}{ax} \PY{o}{=} \PY{n}{sns}\PY{o}{.}\PY{n}{scatterplot}\PY{p}{(}\PY{n}{x}\PY{o}{=}\PY{l+s+s1}{\PYZsq{}}\PY{l+s+s1}{distance}\PY{l+s+s1}{\PYZsq{}}\PY{p}{,} \PY{n}{y}\PY{o}{=}\PY{l+s+s1}{\PYZsq{}}\PY{l+s+s1}{success\PYZus{}prob}\PY{l+s+s1}{\PYZsq{}}\PY{p}{,} \PY{n}{data}\PY{o}{=}\PY{n}{df}\PY{p}{,} \PY{n}{label}\PY{o}{=}\PY{l+s+s1}{\PYZsq{}}\PY{l+s+s1}{Old Dataset}\PY{l+s+s1}{\PYZsq{}}\PY{p}{,} \PY{n}{s}\PY{o}{=}\PY{l+m+mi}{200}\PY{p}{)}
\PY{n}{sns}\PY{o}{.}\PY{n}{scatterplot}\PY{p}{(}\PY{n}{x}\PY{o}{=}\PY{l+s+s1}{\PYZsq{}}\PY{l+s+s1}{distance}\PY{l+s+s1}{\PYZsq{}}\PY{p}{,} \PY{n}{y}\PY{o}{=}\PY{l+s+s1}{\PYZsq{}}\PY{l+s+s1}{success\PYZus{}prob}\PY{l+s+s1}{\PYZsq{}}\PY{p}{,} \PY{n}{data}\PY{o}{=}\PY{n}{new\PYZus{}df}\PY{p}{,}\PY{n}{label}\PY{o}{=}\PY{l+s+s1}{\PYZsq{}}\PY{l+s+s1}{New Dataset}\PY{l+s+s1}{\PYZsq{}}\PY{p}{,} \PY{n}{s}\PY{o}{=}\PY{l+m+mi}{200}\PY{p}{,} \PY{n}{ax}\PY{o}{=}\PY{n}{ax}\PY{p}{)}
\PY{n}{sns}\PY{o}{.}\PY{n}{scatterplot}\PY{p}{(}\PY{n}{x}\PY{o}{=}\PY{l+s+s1}{\PYZsq{}}\PY{l+s+s1}{distance}\PY{l+s+s1}{\PYZsq{}}\PY{p}{,} \PY{n}{y}\PY{o}{=}\PY{n}{new\PYZus{}geo\PYZus{}model\PYZus{}prob}\PY{p}{,} \PY{n}{data}\PY{o}{=}\PY{n}{new\PYZus{}df}\PY{p}{,} \PY{n}{label}\PY{o}{=}\PY{l+s+s1}{\PYZsq{}}\PY{l+s+s1}{Geometry based Model }\PY{l+s+s1}{\PYZsq{}}\PY{p}{,}\PY{n}{ax}\PY{o}{=}\PY{n}{ax}\PY{p}{,} \PY{n}{color}\PY{o}{=}\PY{l+s+s1}{\PYZsq{}}\PY{l+s+s1}{red}\PY{l+s+s1}{\PYZsq{}}\PY{p}{,} \PY{n}{s}\PY{o}{=}\PY{l+m+mi}{100}\PY{p}{)}
\PY{n}{ax}\PY{o}{.}\PY{n}{set}\PY{p}{(}
    \PY{n}{xlabel}\PY{o}{=}\PY{l+s+s1}{\PYZsq{}}\PY{l+s+s1}{Distance from hole(ft)}\PY{l+s+s1}{\PYZsq{}}\PY{p}{,} 
    \PY{n}{ylabel}\PY{o}{=}\PY{l+s+s1}{\PYZsq{}}\PY{l+s+s1}{Probability of Success}\PY{l+s+s1}{\PYZsq{}}
\PY{p}{)}
\PY{n}{plt}\PY{o}{.}\PY{n}{setp}\PY{p}{(}\PY{n}{ax}\PY{o}{.}\PY{n}{get\PYZus{}legend}\PY{p}{(}\PY{p}{)}\PY{o}{.}\PY{n}{get\PYZus{}texts}\PY{p}{(}\PY{p}{)}\PY{p}{,} \PY{n}{fontsize}\PY{o}{=}\PY{l+s+s1}{\PYZsq{}}\PY{l+s+s1}{25}\PY{l+s+s1}{\PYZsq{}}\PY{p}{)}
\end{Verbatim}
\end{tcolorbox}

            \begin{tcolorbox}[breakable, size=fbox, boxrule=.5pt, pad at break*=1mm, opacityfill=0]
\prompt{Out}{outcolor}{94}{\boxspacing}
\begin{Verbatim}[commandchars=\\\{\}]
[None, None, None, None, None, None]
\end{Verbatim}
\end{tcolorbox}
        
    \begin{center}
    \adjustimage{max size={0.9\linewidth}{0.9\paperheight}}{output_56_1.png}
    \end{center}
    { \hspace*{\fill} \\}
    
    We can see:

\begin{itemize}
\tightlist
\item
  Success rate is similar in the 0-20 feet range for both datasets.
\item
  Beyond 20 ft, success rate is lower than expected. These attempts are
  more difficult, even after we have accounted for increased angular
  precision.
\end{itemize}

    \subsection{Moar features!}\label{moar-features}

    To get the ball in, along with the angle, we should also need to take
into account if the ball was hit \textbf{hard enough}.

From Colin Caroll's Blog, we have the following: \textgreater{} Mark
Broadie made the following assumptions - If a putt goes short or more
than 3 feet past the hole, it will not go in. - Golfers aim for 1 foot
past the hole - The distance the ball goes, \(u\), is distributed as:
\[ u \sim \mathcal{N}\left(1 + \text{distance}, \sigma_{\text{distance}} (1 + \text{distance})\right), \]
where we will learn \(\sigma_{\text{distance}}\).

After working through the geometry and algebra, we get:

\[P(\text{Good shot}) = \bigg(2\phi\big(\frac{sin^{-1}(\frac{R-r}{x})}{\sigma_{angle}}\big)-1\bigg)\bigg(\phi\bigg(\frac{2}{(x+1)\sigma_{distance}}\bigg) - \phi\bigg(\frac{-1}{(x+1)\sigma_{distance}}\bigg)\bigg)\]

    Let's write this down in PyMC3

    \begin{tcolorbox}[breakable, size=fbox, boxrule=1pt, pad at break*=1mm,colback=cellbackground, colframe=cellborder]
\prompt{In}{incolor}{107}{\boxspacing}
\begin{Verbatim}[commandchars=\\\{\}]
\PY{n}{OVERSHOT} \PY{o}{=} \PY{l+m+mf}{1.0}
\PY{n}{DISTANCE\PYZus{}TOLERANCE} \PY{o}{=} \PY{l+m+mf}{3.0}
\PY{n}{distances} \PY{o}{=} \PY{n}{new\PYZus{}df}\PY{o}{.}\PY{n}{distance}\PY{o}{.}\PY{n}{values}
\PY{k}{with} \PY{n}{pm}\PY{o}{.}\PY{n}{Model}\PY{p}{(}\PY{p}{)} \PY{k}{as} \PY{n}{model}\PY{p}{:}
    \PY{n}{angle\PYZus{}of\PYZus{}shot\PYZus{}radians} \PY{o}{=} \PY{n}{pm}\PY{o}{.}\PY{n}{HalfNormal}\PY{p}{(}\PY{l+s+s1}{\PYZsq{}}\PY{l+s+s1}{angle\PYZus{}of\PYZus{}shot\PYZus{}radians}\PY{l+s+s1}{\PYZsq{}}\PY{p}{)}
    \PY{n}{angle\PYZus{}of\PYZus{}shot\PYZus{}degrees} \PY{o}{=} \PY{n}{pm}\PY{o}{.}\PY{n}{Deterministic}\PY{p}{(}
        \PY{l+s+s1}{\PYZsq{}}\PY{l+s+s1}{angle\PYZus{}of\PYZus{}shot\PYZus{}degrees}\PY{l+s+s1}{\PYZsq{}}\PY{p}{,}
        \PY{p}{(}\PY{n}{angle\PYZus{}of\PYZus{}shot\PYZus{}radians} \PY{o}{*} \PY{l+m+mf}{180.0}\PY{p}{)} \PY{o}{/} \PY{n}{np}\PY{o}{.}\PY{n}{pi}
    \PY{p}{)}
    
    \PY{n}{variance\PYZus{}of\PYZus{}distance} \PY{o}{=} \PY{n}{pm}\PY{o}{.}\PY{n}{HalfNormal}\PY{p}{(}\PY{l+s+s1}{\PYZsq{}}\PY{l+s+s1}{variance\PYZus{}of\PYZus{}distance}\PY{l+s+s1}{\PYZsq{}}\PY{p}{)}
    \PY{n}{p\PYZus{}good\PYZus{}angle} \PY{o}{=} \PY{n}{pm}\PY{o}{.}\PY{n}{Deterministic}\PY{p}{(}
        \PY{l+s+s1}{\PYZsq{}}\PY{l+s+s1}{p\PYZus{}good\PYZus{}angle}\PY{l+s+s1}{\PYZsq{}}\PY{p}{,}
        \PY{l+m+mi}{2} \PY{o}{*} \PY{n}{calculate\PYZus{}phi}\PY{p}{(}
                \PY{n}{tt}\PY{o}{.}\PY{n}{arcsin}\PY{p}{(}
                    \PY{p}{(}\PY{n}{cup\PYZus{}radius} \PY{o}{\PYZhy{}} \PY{n}{ball\PYZus{}radius}\PY{p}{)}\PY{o}{/} \PY{n}{distances}
                \PY{p}{)} \PY{o}{/} \PY{n}{angle\PYZus{}of\PYZus{}shot\PYZus{}radians}
            \PY{p}{)}
        \PY{p}{)} \PY{o}{\PYZhy{}} \PY{l+m+mi}{1}
    \PY{n}{p\PYZus{}good\PYZus{}distance} \PY{o}{=} \PY{n}{pm}\PY{o}{.}\PY{n}{Deterministic}\PY{p}{(}
        \PY{l+s+s1}{\PYZsq{}}\PY{l+s+s1}{p\PYZus{}good\PYZus{}distance}\PY{l+s+s1}{\PYZsq{}}\PY{p}{,}
        \PY{n}{calculate\PYZus{}phi}\PY{p}{(}
            \PY{p}{(}\PY{n}{DISTANCE\PYZus{}TOLERANCE} \PY{o}{\PYZhy{}} \PY{n}{OVERSHOT}\PY{p}{)} \PY{o}{/} \PY{p}{(}\PY{p}{(}\PY{n}{distances} \PY{o}{+} \PY{n}{OVERSHOT}\PY{p}{)} \PY{o}{*} \PY{n}{variance\PYZus{}of\PYZus{}distance}\PY{p}{)}\PY{p}{)} 
        \PY{o}{\PYZhy{}} \PY{n}{calculate\PYZus{}phi}\PY{p}{(}
            \PY{o}{\PYZhy{}}\PY{n}{OVERSHOT} \PY{o}{/} \PY{p}{(}\PY{p}{(}\PY{n}{distances} \PY{o}{+} \PY{n}{OVERSHOT}\PY{p}{)} \PY{o}{*} \PY{n}{variance\PYZus{}of\PYZus{}distance}\PY{p}{)}\PY{p}{)}

    \PY{p}{)}
    \PY{n}{p\PYZus{}success} \PY{o}{=} \PY{n}{pm}\PY{o}{.}\PY{n}{Binomial}\PY{p}{(}
        \PY{l+s+s1}{\PYZsq{}}\PY{l+s+s1}{p\PYZus{}success}\PY{l+s+s1}{\PYZsq{}}\PY{p}{,}
        \PY{n}{n}\PY{o}{=}\PY{n}{new\PYZus{}df}\PY{o}{.}\PY{n}{tries}\PY{p}{,} 
        \PY{n}{p}\PY{o}{=}\PY{n}{p\PYZus{}good\PYZus{}angle} \PY{o}{*} \PY{n}{p\PYZus{}good\PYZus{}distance}\PY{p}{,} 
        \PY{n}{observed}\PY{o}{=}\PY{n}{new\PYZus{}df}\PY{o}{.}\PY{n}{success\PYZus{}count}
    \PY{p}{)}
    
\end{Verbatim}
\end{tcolorbox}

    \begin{tcolorbox}[breakable, size=fbox, boxrule=1pt, pad at break*=1mm,colback=cellbackground, colframe=cellborder]
\prompt{In}{incolor}{108}{\boxspacing}
\begin{Verbatim}[commandchars=\\\{\}]
\PY{n}{pm}\PY{o}{.}\PY{n}{model\PYZus{}to\PYZus{}graphviz}\PY{p}{(}\PY{n}{model}\PY{p}{)}
\end{Verbatim}
\end{tcolorbox}
 
            
\prompt{Out}{outcolor}{108}{}
    
    \begin{center}
    \adjustimage{max size={0.9\linewidth}{0.9\paperheight}}{output_62_0.pdf}
    \end{center}
    { \hspace*{\fill} \\}
    

    \begin{tcolorbox}[breakable, size=fbox, boxrule=1pt, pad at break*=1mm,colback=cellbackground, colframe=cellborder]
\prompt{In}{incolor}{109}{\boxspacing}
\begin{Verbatim}[commandchars=\\\{\}]
\PY{k}{with} \PY{n}{model}\PY{p}{:}
    \PY{n}{trace} \PY{o}{=} \PY{n}{pm}\PY{o}{.}\PY{n}{sample}\PY{p}{(}\PY{l+m+mi}{1000}\PY{p}{,} \PY{n}{tune}\PY{o}{=}\PY{l+m+mi}{1000}\PY{p}{,} \PY{n}{chains}\PY{o}{=}\PY{l+m+mi}{4}\PY{p}{)}
\end{Verbatim}
\end{tcolorbox}

    \begin{Verbatim}[commandchars=\\\{\}]
Auto-assigning NUTS sampler{\ldots}
Initializing NUTS using jitter+adapt\_diag{\ldots}
Multiprocess sampling (4 chains in 2 jobs)
NUTS: [variance\_of\_distance, angle\_of\_shot\_radians]
Sampling 4 chains, 0 divergences: 100\%|██████████| 8000/8000 [01:57<00:00,
67.90draws/s]
The acceptance probability does not match the target. It is 0.9997747107105002,
but should be close to 0.8. Try to increase the number of tuning steps.
The number of effective samples is smaller than 25\% for some parameters.
    \end{Verbatim}

    \begin{tcolorbox}[breakable, size=fbox, boxrule=1pt, pad at break*=1mm,colback=cellbackground, colframe=cellborder]
\prompt{In}{incolor}{110}{\boxspacing}
\begin{Verbatim}[commandchars=\\\{\}]
\PY{n}{pm}\PY{o}{.}\PY{n}{summary}\PY{p}{(}\PY{n}{trace}\PY{p}{)}\PY{o}{.}\PY{n}{head}\PY{p}{(}\PY{l+m+mi}{3}\PY{p}{)}
\end{Verbatim}
\end{tcolorbox}

            \begin{tcolorbox}[breakable, size=fbox, boxrule=.5pt, pad at break*=1mm, opacityfill=0]
\prompt{Out}{outcolor}{110}{\boxspacing}
\begin{Verbatim}[commandchars=\\\{\}]
                        mean     sd  hpd\_3\%  hpd\_97\%  mcse\_mean  mcse\_sd  \textbackslash{}
angle\_of\_shot\_radians  0.013  0.000   0.013    0.013        0.0      0.0
angle\_of\_shot\_degrees  0.761  0.003   0.755    0.768        0.0      0.0
variance\_of\_distance   0.137  0.001   0.136    0.138        0.0      0.0

                       ess\_mean  ess\_sd  ess\_bulk  ess\_tail  r\_hat
angle\_of\_shot\_radians     810.0   810.0     818.0    1268.0   1.01
angle\_of\_shot\_degrees     810.0   810.0     818.0    1268.0   1.01
variance\_of\_distance      935.0   935.0     936.0    1361.0   1.01
\end{Verbatim}
\end{tcolorbox}
        
    \begin{tcolorbox}[breakable, size=fbox, boxrule=1pt, pad at break*=1mm,colback=cellbackground, colframe=cellborder]
\prompt{In}{incolor}{128}{\boxspacing}
\begin{Verbatim}[commandchars=\\\{\}]
\PY{n}{pm}\PY{o}{.}\PY{n}{plot\PYZus{}posterior}\PY{p}{(}\PY{n}{trace}\PY{p}{[}\PY{l+s+s1}{\PYZsq{}}\PY{l+s+s1}{variance\PYZus{}of\PYZus{}distance}\PY{l+s+s1}{\PYZsq{}}\PY{p}{]}\PY{p}{)}
\end{Verbatim}
\end{tcolorbox}

            \begin{tcolorbox}[breakable, size=fbox, boxrule=.5pt, pad at break*=1mm, opacityfill=0]
\prompt{Out}{outcolor}{128}{\boxspacing}
\begin{Verbatim}[commandchars=\\\{\}]
array([<matplotlib.axes.\_subplots.AxesSubplot object at 0x7f7f112347f0>],
      dtype=object)
\end{Verbatim}
\end{tcolorbox}
        
    \begin{center}
    \adjustimage{max size={0.9\linewidth}{0.9\paperheight}}{output_65_1.png}
    \end{center}
    { \hspace*{\fill} \\}
    
    \begin{tcolorbox}[breakable, size=fbox, boxrule=1pt, pad at break*=1mm,colback=cellbackground, colframe=cellborder]
\prompt{In}{incolor}{113}{\boxspacing}
\begin{Verbatim}[commandchars=\\\{\}]
\PY{k}{with} \PY{n}{model}\PY{p}{:}
    \PY{n}{distance\PYZus{}posterior} \PY{o}{=} \PY{n}{pm}\PY{o}{.}\PY{n}{sample\PYZus{}posterior\PYZus{}predictive}\PY{p}{(}\PY{n}{trace}\PY{p}{)}
\end{Verbatim}
\end{tcolorbox}

    \begin{Verbatim}[commandchars=\\\{\}]
100\%|██████████| 4000/4000 [00:06<00:00, 627.04it/s]
    \end{Verbatim}

    \begin{tcolorbox}[breakable, size=fbox, boxrule=1pt, pad at break*=1mm,colback=cellbackground, colframe=cellborder]
\prompt{In}{incolor}{150}{\boxspacing}
\begin{Verbatim}[commandchars=\\\{\}]
\PY{k}{def} \PY{n+nf}{calculate\PYZus{}prob\PYZus{}distance}\PY{p}{(}\PY{n}{angle}\PY{p}{,} \PY{n}{distance}\PY{p}{,} \PY{n}{ls}\PY{p}{)}\PY{p}{:}
    \PY{l+s+sd}{\PYZdq{}\PYZdq{}\PYZdq{}}
\PY{l+s+sd}{    Calculate the probability the ball will land inside the hole}
\PY{l+s+sd}{    given the variance in angle and distance.}
\PY{l+s+sd}{    }
\PY{l+s+sd}{    NOTE: Adapted from Colin Carroll\PYZsq{}s Blog.}
\PY{l+s+sd}{    \PYZdq{}\PYZdq{}\PYZdq{}}
    \PY{n}{norm} \PY{o}{=} \PY{n}{scipy}\PY{o}{.}\PY{n}{stats}\PY{o}{.}\PY{n}{norm}\PY{p}{(}\PY{l+m+mi}{0}\PY{p}{,} \PY{l+m+mi}{1}\PY{p}{)}
    \PY{n}{prob\PYZus{}angle} \PY{o}{=} \PY{l+m+mi}{2} \PY{o}{*} \PY{n}{norm}\PY{o}{.}\PY{n}{cdf}\PY{p}{(}
        \PY{n}{np}\PY{o}{.}\PY{n}{arcsin}\PY{p}{(}\PY{p}{(}\PY{n}{cup\PYZus{}radius} \PY{o}{\PYZhy{}} \PY{n}{ball\PYZus{}radius}\PY{p}{)} \PY{o}{/} \PY{n}{ls}\PY{p}{)} \PY{o}{/} \PY{n}{angle}\PY{p}{)} \PY{o}{\PYZhy{}} \PY{l+m+mi}{1}
    \PY{n}{prob\PYZus{}distance\PYZus{}one} \PY{o}{=} \PY{n}{norm}\PY{o}{.}\PY{n}{cdf}\PY{p}{(}
        \PY{p}{(}\PY{n}{DISTANCE\PYZus{}TOLERANCE} \PY{o}{\PYZhy{}} \PY{n}{OVERSHOT}\PY{p}{)} \PY{o}{/} \PY{p}{(}\PY{p}{(}\PY{n}{ls} \PY{o}{+} \PY{n}{OVERSHOT}\PY{p}{)} \PY{o}{*} \PY{n}{distance}\PY{p}{)}
    \PY{p}{)}
    \PY{n}{prob\PYZus{}distance\PYZus{}two} \PY{o}{=} \PY{n}{norm}\PY{o}{.}\PY{n}{cdf}\PY{p}{(}\PY{o}{\PYZhy{}}\PY{n}{OVERSHOT} \PY{o}{/} \PY{p}{(}\PY{p}{(}\PY{n}{ls} \PY{o}{+} \PY{n}{OVERSHOT}\PY{p}{)} \PY{o}{*} \PY{n}{distance}\PY{p}{)}\PY{p}{)}
    \PY{n}{prob\PYZus{}distance} \PY{o}{=} \PY{n}{prob\PYZus{}distance\PYZus{}one} \PY{o}{\PYZhy{}} \PY{n}{prob\PYZus{}distance\PYZus{}two}
    
    \PY{k}{return} \PY{n}{prob\PYZus{}angle} \PY{o}{*} \PY{n}{prob\PYZus{}distance}
\end{Verbatim}
\end{tcolorbox}

    \begin{tcolorbox}[breakable, size=fbox, boxrule=1pt, pad at break*=1mm,colback=cellbackground, colframe=cellborder]
\prompt{In}{incolor}{137}{\boxspacing}
\begin{Verbatim}[commandchars=\\\{\}]
\PY{n}{distance\PYZus{}model\PYZus{}prob} \PY{o}{=} \PY{p}{[}\PY{p}{]}
\PY{k}{for} \PY{n}{point} \PY{o+ow}{in} \PY{n}{ls}\PY{p}{:}
    \PY{n}{prob} \PY{o}{=} 
    \PY{n}{distance\PYZus{}model\PYZus{}prob}\PY{o}{.}\PY{n}{append}\PY{p}{(}\PY{n}{prob}\PY{p}{)}
    
\end{Verbatim}
\end{tcolorbox}

    \begin{tcolorbox}[breakable, size=fbox, boxrule=1pt, pad at break*=1mm,colback=cellbackground, colframe=cellborder]
\prompt{In}{incolor}{142}{\boxspacing}
\begin{Verbatim}[commandchars=\\\{\}]
\PY{n}{ls} \PY{o}{=} \PY{n}{np}\PY{o}{.}\PY{n}{linspace}\PY{p}{(}\PY{l+m+mi}{0}\PY{p}{,} \PY{n}{new\PYZus{}df}\PY{o}{.}\PY{n}{distance}\PY{o}{.}\PY{n}{max}\PY{p}{(}\PY{p}{)}\PY{p}{,} \PY{l+m+mi}{200}\PY{p}{)}
\end{Verbatim}
\end{tcolorbox}

    \begin{tcolorbox}[breakable, size=fbox, boxrule=1pt, pad at break*=1mm,colback=cellbackground, colframe=cellborder]
\prompt{In}{incolor}{156}{\boxspacing}
\begin{Verbatim}[commandchars=\\\{\}]
\PY{n}{new\PYZus{}df}\PY{p}{[}\PY{l+s+s1}{\PYZsq{}}\PY{l+s+s1}{success\PYZus{}prob}\PY{l+s+s1}{\PYZsq{}}\PY{p}{]} \PY{o}{=} \PY{n}{new\PYZus{}df}\PY{o}{.}\PY{n}{success\PYZus{}count} \PY{o}{/} \PY{n}{new\PYZus{}df}\PY{o}{.}\PY{n}{tries}
\PY{n}{sns}\PY{o}{.}\PY{n}{set}\PY{p}{(}\PY{p}{)}
\PY{n}{plt}\PY{o}{.}\PY{n}{figure}\PY{p}{(}\PY{n}{figsize}\PY{o}{=}\PY{p}{(}\PY{l+m+mi}{16}\PY{p}{,} \PY{l+m+mi}{6}\PY{p}{)}\PY{p}{)}
\PY{n}{ax} \PY{o}{=} \PY{n}{sns}\PY{o}{.}\PY{n}{scatterplot}\PY{p}{(}
    \PY{n}{x}\PY{o}{=}\PY{l+s+s1}{\PYZsq{}}\PY{l+s+s1}{distance}\PY{l+s+s1}{\PYZsq{}}\PY{p}{,} 
    \PY{n}{y}\PY{o}{=}\PY{l+s+s1}{\PYZsq{}}\PY{l+s+s1}{success\PYZus{}prob}\PY{l+s+s1}{\PYZsq{}}\PY{p}{,}
    \PY{n}{data}\PY{o}{=}\PY{n}{new\PYZus{}df}\PY{p}{,}
    \PY{n}{label}\PY{o}{=}\PY{l+s+s1}{\PYZsq{}}\PY{l+s+s1}{Actual}\PY{l+s+s1}{\PYZsq{}}\PY{p}{,} 
    \PY{n}{s}\PY{o}{=}\PY{l+m+mi}{200}
\PY{p}{)}
\PY{n}{sns}\PY{o}{.}\PY{n}{scatterplot}\PY{p}{(}
    \PY{n}{x}\PY{o}{=}\PY{l+s+s1}{\PYZsq{}}\PY{l+s+s1}{distance}\PY{l+s+s1}{\PYZsq{}}\PY{p}{,} 
    \PY{n}{y}\PY{o}{=}\PY{n}{new\PYZus{}geo\PYZus{}model\PYZus{}prob}\PY{p}{,} 
    \PY{n}{data}\PY{o}{=}\PY{n}{new\PYZus{}df}\PY{p}{,} 
    \PY{n}{label}\PY{o}{=}\PY{l+s+s1}{\PYZsq{}}\PY{l+s+s1}{Angle only Model}\PY{l+s+s1}{\PYZsq{}}\PY{p}{,}
    \PY{n}{ax}\PY{o}{=}\PY{n}{ax}\PY{p}{,} 
    \PY{n}{color}\PY{o}{=}\PY{l+s+s1}{\PYZsq{}}\PY{l+s+s1}{red}\PY{l+s+s1}{\PYZsq{}}\PY{p}{,} 
    \PY{n}{s}\PY{o}{=}\PY{l+m+mi}{100}
\PY{p}{)}

\PY{n}{sns}\PY{o}{.}\PY{n}{scatterplot}\PY{p}{(}
    \PY{n}{x}\PY{o}{=}\PY{l+s+s1}{\PYZsq{}}\PY{l+s+s1}{distance}\PY{l+s+s1}{\PYZsq{}}\PY{p}{,} 
    \PY{n}{y}\PY{o}{=}\PY{n}{calculate\PYZus{}prob\PYZus{}distance}\PY{p}{(}
        \PY{n}{trace}\PY{p}{[}\PY{l+s+s1}{\PYZsq{}}\PY{l+s+s1}{angle\PYZus{}of\PYZus{}shot\PYZus{}radians}\PY{l+s+s1}{\PYZsq{}}\PY{p}{]}\PY{o}{.}\PY{n}{mean}\PY{p}{(}\PY{p}{)}\PY{p}{,} 
        \PY{n}{trace}\PY{p}{[}\PY{l+s+s1}{\PYZsq{}}\PY{l+s+s1}{variance\PYZus{}of\PYZus{}distance}\PY{l+s+s1}{\PYZsq{}}\PY{p}{]}\PY{o}{.}\PY{n}{mean}\PY{p}{(}\PY{p}{)}\PY{p}{,}
        \PY{n}{new\PYZus{}df}\PY{o}{.}\PY{n}{distance}
    \PY{p}{)}\PY{p}{,} 
    \PY{n}{data}\PY{o}{=}\PY{n}{new\PYZus{}df}\PY{p}{,} 
    \PY{n}{label}\PY{o}{=}\PY{l+s+s1}{\PYZsq{}}\PY{l+s+s1}{Distance + Angle based Model }\PY{l+s+s1}{\PYZsq{}}\PY{p}{,}
    \PY{n}{ax}\PY{o}{=}\PY{n}{ax}\PY{p}{,} 
    \PY{n}{color}\PY{o}{=}\PY{l+s+s1}{\PYZsq{}}\PY{l+s+s1}{black}\PY{l+s+s1}{\PYZsq{}}\PY{p}{,} 
    \PY{n}{s}\PY{o}{=}\PY{l+m+mi}{100}
\PY{p}{)}
\PY{n}{ax}\PY{o}{.}\PY{n}{set}\PY{p}{(}
    \PY{n}{xlabel}\PY{o}{=}\PY{l+s+s1}{\PYZsq{}}\PY{l+s+s1}{Distance from hole(ft)}\PY{l+s+s1}{\PYZsq{}}\PY{p}{,} 
    \PY{n}{ylabel}\PY{o}{=}\PY{l+s+s1}{\PYZsq{}}\PY{l+s+s1}{Probability of Success}\PY{l+s+s1}{\PYZsq{}}
\PY{p}{)}

\PY{n}{plt}\PY{o}{.}\PY{n}{setp}\PY{p}{(}\PY{n}{ax}\PY{o}{.}\PY{n}{get\PYZus{}legend}\PY{p}{(}\PY{p}{)}\PY{o}{.}\PY{n}{get\PYZus{}texts}\PY{p}{(}\PY{p}{)}\PY{p}{,} \PY{n}{fontsize}\PY{o}{=}\PY{l+s+s1}{\PYZsq{}}\PY{l+s+s1}{25}\PY{l+s+s1}{\PYZsq{}}\PY{p}{)}
\end{Verbatim}
\end{tcolorbox}

            \begin{tcolorbox}[breakable, size=fbox, boxrule=.5pt, pad at break*=1mm, opacityfill=0]
\prompt{Out}{outcolor}{156}{\boxspacing}
\begin{Verbatim}[commandchars=\\\{\}]
[None, None, None, None, None, None]
\end{Verbatim}
\end{tcolorbox}
        
    \begin{center}
    \adjustimage{max size={0.9\linewidth}{0.9\paperheight}}{output_70_1.png}
    \end{center}
    { \hspace*{\fill} \\}
    
    From the graph, we can conclude that:

\begin{itemize}
\tightlist
\item
  The model is good at distance lower than 10 ft and distances higher
  than 40ft.
\item
  There is some mismatch between 10ft to 40ft, but overall this is a
  good fit.
\end{itemize}

    \subsection{What's the point?}\label{whats-the-point}

    Using Bayesian analysis, we want to be able to quantify the unvertainity
with each of our predictions. Since each prediction is a distribution,
we can utilize this to see where the putts will fall if they do not fall
in the hole.

    \begin{tcolorbox}[breakable, size=fbox, boxrule=1pt, pad at break*=1mm,colback=cellbackground, colframe=cellborder]
\prompt{In}{incolor}{166}{\boxspacing}
\begin{Verbatim}[commandchars=\\\{\}]
\PY{k}{def} \PY{n+nf}{simulate\PYZus{}from\PYZus{}distance}\PY{p}{(}\PY{n}{trace}\PY{p}{,} \PY{n}{distance\PYZus{}to\PYZus{}hole}\PY{p}{,} \PY{n}{trials}\PY{o}{=}\PY{l+m+mi}{10\PYZus{}000}\PY{p}{)}\PY{p}{:}
    \PY{n}{n\PYZus{}samples} \PY{o}{=} \PY{n}{trace}\PY{p}{[}\PY{l+s+s1}{\PYZsq{}}\PY{l+s+s1}{angle\PYZus{}of\PYZus{}shot\PYZus{}radians}\PY{l+s+s1}{\PYZsq{}}\PY{p}{]}\PY{o}{.}\PY{n}{shape}\PY{p}{[}\PY{l+m+mi}{0}\PY{p}{]}

    \PY{n}{idxs} \PY{o}{=} \PY{n}{np}\PY{o}{.}\PY{n}{random}\PY{o}{.}\PY{n}{randint}\PY{p}{(}\PY{l+m+mi}{0}\PY{p}{,} \PY{n}{n\PYZus{}samples}\PY{p}{,} \PY{n}{trials}\PY{p}{)}
    \PY{n}{variance\PYZus{}of\PYZus{}shot} \PY{o}{=} \PY{n}{trace}\PY{p}{[}\PY{l+s+s1}{\PYZsq{}}\PY{l+s+s1}{angle\PYZus{}of\PYZus{}shot\PYZus{}radians}\PY{l+s+s1}{\PYZsq{}}\PY{p}{]}\PY{p}{[}\PY{n}{idxs}\PY{p}{]}
    \PY{n}{variance\PYZus{}of\PYZus{}distance} \PY{o}{=} \PY{n}{trace}\PY{p}{[}\PY{l+s+s1}{\PYZsq{}}\PY{l+s+s1}{variance\PYZus{}of\PYZus{}distance}\PY{l+s+s1}{\PYZsq{}}\PY{p}{]}\PY{p}{[}\PY{n}{idxs}\PY{p}{]}

    \PY{n}{theta} \PY{o}{=} \PY{n}{np}\PY{o}{.}\PY{n}{random}\PY{o}{.}\PY{n}{normal}\PY{p}{(}\PY{l+m+mi}{0}\PY{p}{,} \PY{n}{variance\PYZus{}of\PYZus{}shot}\PY{p}{)}
    \PY{n}{distance} \PY{o}{=} \PY{n}{np}\PY{o}{.}\PY{n}{random}\PY{o}{.}\PY{n}{normal}\PY{p}{(}\PY{n}{distance\PYZus{}to\PYZus{}hole} \PY{o}{+} \PY{n}{OVERSHOT}\PY{p}{,} \PY{p}{(}\PY{n}{distance\PYZus{}to\PYZus{}hole} \PY{o}{+} \PY{n}{OVERSHOT}\PY{p}{)} \PY{o}{*} \PY{n}{variance\PYZus{}of\PYZus{}distance}\PY{p}{)}

    \PY{n}{final\PYZus{}position} \PY{o}{=} \PY{n}{np}\PY{o}{.}\PY{n}{array}\PY{p}{(}\PY{p}{[}\PY{n}{distance} \PY{o}{*} \PY{n}{np}\PY{o}{.}\PY{n}{cos}\PY{p}{(}\PY{n}{theta}\PY{p}{)}\PY{p}{,} \PY{n}{distance} \PY{o}{*} \PY{n}{np}\PY{o}{.}\PY{n}{sin}\PY{p}{(}\PY{n}{theta}\PY{p}{)}\PY{p}{]}\PY{p}{)}

    \PY{n}{made\PYZus{}it} \PY{o}{=} \PY{n}{np}\PY{o}{.}\PY{n}{abs}\PY{p}{(}\PY{n}{theta}\PY{p}{)} \PY{o}{\PYZlt{}} \PY{n}{np}\PY{o}{.}\PY{n}{arcsin}\PY{p}{(}\PY{p}{(}\PY{n}{cup\PYZus{}radius} \PY{o}{\PYZhy{}} \PY{n}{ball\PYZus{}radius}\PY{p}{)} \PY{o}{/} \PY{n}{distance\PYZus{}to\PYZus{}hole}\PY{p}{)}
    \PY{n}{made\PYZus{}it} \PY{o}{=} \PY{n}{made\PYZus{}it} \PY{o}{*} \PY{p}{(}\PY{n}{final\PYZus{}position}\PY{p}{[}\PY{l+m+mi}{0}\PY{p}{]} \PY{o}{\PYZgt{}} \PY{n}{distance\PYZus{}to\PYZus{}hole}\PY{p}{)} \PY{o}{*} \PY{p}{(}\PY{n}{final\PYZus{}position}\PY{p}{[}\PY{l+m+mi}{0}\PY{p}{]} \PY{o}{\PYZlt{}} \PY{n}{distance\PYZus{}to\PYZus{}hole} \PY{o}{+} \PY{n}{DISTANCE\PYZus{}TOLERANCE}\PY{p}{)}
    
    \PY{n}{\PYZus{}}\PY{p}{,} \PY{n}{ax} \PY{o}{=} \PY{n}{plt}\PY{o}{.}\PY{n}{subplots}\PY{p}{(}\PY{p}{)}

    \PY{n}{ax}\PY{o}{.}\PY{n}{plot}\PY{p}{(}\PY{l+m+mi}{0}\PY{p}{,} \PY{l+m+mi}{0}\PY{p}{,} \PY{l+s+s1}{\PYZsq{}}\PY{l+s+s1}{k.}\PY{l+s+s1}{\PYZsq{}}\PY{p}{,} \PY{n}{lw}\PY{o}{=}\PY{l+m+mi}{1}\PY{p}{,} \PY{n}{mfc}\PY{o}{=}\PY{l+s+s1}{\PYZsq{}}\PY{l+s+s1}{black}\PY{l+s+s1}{\PYZsq{}}\PY{p}{,} \PY{n}{ms}\PY{o}{=}\PY{l+m+mi}{150} \PY{o}{/} \PY{n}{distance\PYZus{}to\PYZus{}hole}\PY{p}{)}
    \PY{n}{ax}\PY{o}{.}\PY{n}{plot}\PY{p}{(}\PY{o}{*}\PY{n}{final\PYZus{}position}\PY{p}{[}\PY{p}{:}\PY{p}{,} \PY{o}{\PYZti{}}\PY{n}{made\PYZus{}it}\PY{p}{]}\PY{p}{,} \PY{l+s+s1}{\PYZsq{}}\PY{l+s+s1}{.}\PY{l+s+s1}{\PYZsq{}}\PY{p}{,} \PY{n}{alpha}\PY{o}{=}\PY{l+m+mf}{0.1}\PY{p}{,} \PY{n}{mfc}\PY{o}{=}\PY{l+s+s1}{\PYZsq{}}\PY{l+s+s1}{r}\PY{l+s+s1}{\PYZsq{}}\PY{p}{,} \PY{n}{ms}\PY{o}{=}\PY{l+m+mi}{250} \PY{o}{/} \PY{n}{distance\PYZus{}to\PYZus{}hole}\PY{p}{,} \PY{n}{mew}\PY{o}{=}\PY{l+m+mf}{0.5}\PY{p}{)}
    \PY{n}{ax}\PY{o}{.}\PY{n}{plot}\PY{p}{(}\PY{n}{distance\PYZus{}to\PYZus{}hole}\PY{p}{,} \PY{l+m+mi}{0}\PY{p}{,} \PY{l+s+s1}{\PYZsq{}}\PY{l+s+s1}{ko}\PY{l+s+s1}{\PYZsq{}}\PY{p}{,} \PY{n}{lw}\PY{o}{=}\PY{l+m+mi}{1}\PY{p}{,} \PY{n}{mfc}\PY{o}{=}\PY{l+s+s1}{\PYZsq{}}\PY{l+s+s1}{black}\PY{l+s+s1}{\PYZsq{}}\PY{p}{,} \PY{n}{ms}\PY{o}{=}\PY{l+m+mi}{350} \PY{o}{/} \PY{n}{distance\PYZus{}to\PYZus{}hole}\PY{p}{)}

    \PY{n}{ax}\PY{o}{.}\PY{n}{set\PYZus{}facecolor}\PY{p}{(}\PY{l+s+s2}{\PYZdq{}}\PY{l+s+s2}{\PYZsh{}e6ffdb}\PY{l+s+s2}{\PYZdq{}}\PY{p}{)}
    \PY{n}{ax}\PY{o}{.}\PY{n}{set\PYZus{}title}\PY{p}{(}\PY{l+s+sa}{f}\PY{l+s+s2}{\PYZdq{}}\PY{l+s+s2}{Final position of }\PY{l+s+si}{\PYZob{}trials:,d\PYZcb{}}\PY{l+s+s2}{ putts from }\PY{l+s+si}{\PYZob{}distance\PYZus{}to\PYZus{}hole\PYZcb{}}\PY{l+s+s2}{ft.}\PY{l+s+se}{\PYZbs{}n}\PY{l+s+s2}{(}\PY{l+s+s2}{\PYZob{}}\PY{l+s+s2}{100 * made\PYZus{}it.mean():.1f\PYZcb{}}\PY{l+s+s2}{\PYZpc{}}\PY{l+s+s2}{ made)}\PY{l+s+s2}{\PYZdq{}}\PY{p}{)}
    \PY{k}{return} \PY{n}{ax}

\PY{n}{simulate\PYZus{}from\PYZus{}distance}\PY{p}{(}\PY{n}{trace}\PY{p}{,} \PY{n}{distance\PYZus{}to\PYZus{}hole}\PY{o}{=}\PY{l+m+mi}{10}\PY{p}{)}\PY{p}{;}
\end{Verbatim}
\end{tcolorbox}

    \begin{center}
    \adjustimage{max size={0.9\linewidth}{0.9\paperheight}}{output_74_0.png}
    \end{center}
    { \hspace*{\fill} \\}
    
    \section{Conclusion}\label{conclusion}

    We've just seen how incorporate subjective knowledge in our models and
help them fit cases that are specific to our use-case.

References:

\begin{itemize}
\tightlist
\item
  This is heavily inspired by Colin Caroll's Blog present
  \href{https://nbviewer.jupyter.org/github/pymc-devs/pymc3/blob/master/docs/source/notebooks/putting_workflow.ipynb}{here}
\item
  The crux of this post is based on Dr. Gelman's case study present
  \href{https://mc-stan.org/users/documentation/case-studies/golf.html}{here}.
\end{itemize}

    \begin{tcolorbox}[breakable, size=fbox, boxrule=1pt, pad at break*=1mm,colback=cellbackground, colframe=cellborder]
\prompt{In}{incolor}{ }{\boxspacing}
\begin{Verbatim}[commandchars=\\\{\}]

\end{Verbatim}
\end{tcolorbox}


    % Add a bibliography block to the postdoc
    
    
    
\end{document}
